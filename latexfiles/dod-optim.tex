\chapter[Optimisations]{Optimisations and Implementations}

When optimising software, you have to know what is causing the software to run
slower than you need it to run. We find in most cases, data movement is what
really costs us the most. In the GPU, we find it labelled under fill rate, and
when on CPU, we call it cache-misses. Data movement is where most of the energy
goes when processing data, not in calculating solutions to functions, or from
running an algorithm on the data, but actually the fulfillment of the request
for the data in the first place.  As this is most definitely true about our
current architectures, we find that implicit or calculable information is much
more useful than cached values or explicit state data.

If we start our game development by organising our data in normalised tables,
we have many opportunities for optimisation. Starting with such a problem
agnostic layout, we can pick and choose from tools we've created for other
tasks, at worst elevating the solution to a template or a strategy, before
applying it to both the old and new use cases.

\section{Tables}

To keep things simple, advice from multiple sources indicates that keeping your
data as vectors has a lot of positive benefits. There are reasons to not use
STL, including extended compile and link times, as well as issues with memory
allocations. Whether you use std::vector, or roll your own dynamicly sized
array, it is a good starting place for any future optimisations. Most of the
processing you will do will be transforming one array into another, or
modifying a table in place. In both these cases, a simple array will suffice
for most tasks.

For the benefit of your cache, structs of arrays can be more cache friendly if
the data is not related. It's important to remember that this is only true when
the data is not meant to be accessed all at once, as one advocate of the
data-oriented design movement assumed that structures of arrays were
intrinsically cache friendly, then put the x,y, and z coordinates in separate
arrays of floats. The reason that this is not cache friendly should be
relatively easy to spot. If you need to access the x,y, or z of an element in
an array, then you more than likely need to access the other two axes as well.
This means that for every element he would have been loading three cache-lines
of float data, not one.  This is why it is important to think about where the
data is coming from, how it is related, and how it will be used. Data-oriented
design is not just a set of simple rules to convert from one style to another.

If you use dynamic arrays, and you need to delete elements from them, and these
tables refer to each other through some IDs, then you may need a way to splice
the tables together in order to process them. If the tables are sorted by the
same value, then it can be written out as a simple merge operation, such as in
Listing \ref{lst:joinmerge}.

\begin{lstlisting}[caption=Zipping together multiple tables by merging,float=h,label=lst:joinmerge]
Table<Type1> t1Table;
Table<Type2> t2Table;
Table<Type3> t3Table;

ProcessJoin( Func functionToCall ) {
	TableIterator A = t1Table.begin();
	TableIterator B = t2Table.begin();
	TableIterator C = t3Table.begin();
	while( !A.finished && !B.finished && !C.finished ) {
		if( A == B && B == C ) {
			functionToCall( A, B, C );
			++A; ++B; ++C;
		} else {
			if( A < B || A < C ) ++A;
			if( B < A || B < C ) ++B;
			if( C < A || C < B ) ++C;
		}
	}
}
\end{lstlisting}

This works as long as the == operator knows about the table types and can find
the specific column to check against, and as long as the tables are sorted
based on this same column. But what about the case where the tables are zipped
together without being the sorted by the same columns? For example, if you have
a lot of entities that refer to a modelID, and you have a lot of mesh-texture
combinations that refer to the same modelID, then you will likely need to zip
together the matching rows for the orientation of the entity, the modelID in
the entity render data, and the mesh and texture combinations in the models.
The simplest way to program a solution to this is to loop through each table in
turn looking for matches such as in Listing \ref{lst:joinloop}.

\begin{lstlisting}[caption=Join by looping through all tables,label=lst:joinloop,float=h]
Table<Orientation> oTable;
Table<EntityRenderable> erTable;
Table<MeshAndTexture> modelAssets;

ProcessJoin( Func functionToCall ) {
	TableIterator A = oTable.begin();
	while( !A.finished() ) {
		TableIterator B = erTable.begin();
		while( !B.finished() ) {
			TableIterator C = modelAssets.begin();
			while( !C.finished() ) {
				if( A == B && B == C ) {
					functionToCall( A, B, C );
				}
			 ++C;
			}
		++B;
		}
	++A;
	}
}
\end{lstlisting}

Another thing you have to learn about when working with data that is joined on
different columns is the use of join strategies. In databases, a join strategy
is used to reduce the total number of operations when querying across multiple
tables. When joining tables on a column (or key made up of multiple columns),
you have a number of choices about how you approach the problem. In our trivial
coded attempt you can see we simply iterate over the whole table for each table
involved in the join, which ends up being O($nmo$) or O($n^3$) for rougly same
size tables. This is no good for large tables, but for small ones it's fine.
You have to know your data to decide whether your tables are
big\footnote{dependent on the target hardware, how many rows and columns, and
whether you want the process to run without trashing too much cache} or not. If
your tables are too big to use such a trivial join, then you will need an
alternative strategy.

You can join by iteration, or you can join by lookup\footnote{often a lookup
join is called a join by hash, but as we know our data, we can use better row
search algorithms than a hash when they are available}, or you can even join
once and keep a join cache around.

The first thing could do is use the ability to have tables sorted in multiple
ways at the same time. Though this seems impossible, it's perfectly feasible to
add auxilary data that will allow for traversal of a table in a different
order. We do this the same way databases allow for any number of indexes into a
table. Each index is created and kept up to date as the table is modified. In
our case, we implement each index the way we need to. Maybe some tables are
written to in bursts, and an insertion sort would be slow, it might be better
to sort on first read. In other cases, the sorting might be better done on
write, as the writes are infrequent, or always interleaved with reads.

Concatenation trees provide a quick way to traverse a list. Conc-trees usually
are only minimally slower than a linear array due to the nature of the
structure. A conc-tree is a high level structure that points to a low level
structure, and many elements can pass through a process before the next leaf
needs to be loaded. The code for a conc-tree doesn't remain in memory all the
time like other list handling code, as the list offloads to an array iteration
whenever it is able. This alone means that sparse conc-trees end up spending
little time in their own code, and offer the benefit of not having to rebuild
when an element goes missing from the middle of the array.

In addition to using concatenation trees to provide a standard iterator for a
constantly modified data store, it can also be used as a way of storing
multiple views into data. For example, perhaps there is a set of tables that
are all the same data, and they all need to be processed, but they are stored
as different tables for some reason, such as what team they are on. Using the
same conc-tree code, they can be iterated as a full collection with any code
that accepts a conc-tree instead of an array iterator.

Modifying a class can be difficult, especially at runtime when you can't
affect the offsets in running code without at least stopping the process and
updating the executable. Adding new elements to a class can be very useful, and
in languages that allow it, such as Ruby and Python, and even Javascript, it
can be used as a substitute for virtual functions and compositing. In other
languages we can add new functions, new data, and use them at runtime. In C++,
we cannot change existing classes, unless they are defined by some other
mechanism than the compile time structure syntax. We can add elements if we
define a class by its schema, a data driven representation of the elements in
the class. The benefit of this is that schema can include more, or less, than
normal given a new context. For example, in the case of some merging where the
merging happens on two different axes, there could be two different schema
representing the same class.  The indexes would be different. One schema could
include both indexes, with which it would build up an combination table that
included the first two tables merged, by the first index, but maintaining the
same ordering so that when merging the merged table with the third table to be
merged, the second index can be used to maintain efficient combination.

\begin{verbatim}
A, B, C
B has index(AB), and index(BC)
merge A,B (by index(AB))
AB, C
merge AB, C (by index(BC))
ABC
\end{verbatim}

\section{Transforms}

Taking the concept of schemas another step, a static schema definition can
allow for a different approach to iterators. Instead of iterating over a
container, giving access to an element, a schema iterator can become an
accessor for a set of tables, meaning the merging work can be done during
iteration, generating a context upon which the transform operates. This would
benefit large, complex merges that do little with the data, as there would be
less memory usage creating temporary tables. It would not benefit complex
transforms as it would reduce the likelihood that the next set of data is in
memory ready for the next cycle.

For large jobs, a smarter iterator will help in task stealing, the concept of
taking work away from a process that is already running in order to finish the
job faster. A scheduler, or job management system, built for such situations,
would monitor how tasks were progressing and split remaining work amongst any
idle processors. Sometimes this will happen because other processes took less
time to finish than expected, sometimes because a single task just takes longer
than expected. Whatever the reason, a transform based design makes task
stealing much simpler than the standard sequential model, and provides a
mechanism by which many tasks can be made significantly more parallel.

Another aspect of transforms is the separation of what from how, the separation
of the loading of data to transform from the code that performs the operations
on the data. In some languages, introducing map and reduce is part of the basic
syllabus, in c++, not so much. This is probably because lists aren't part of
the base language, and without that, it's hard to introduce powerful tools that
require an understanding of them. These tools, map and reduce, can be the basis
of a purely transform and flow driven program. Turning a large set of data into
a single result sounds eminently serial, however, as long as one of the steps,
the reduce step, is either associative, or commutative, then you can reduce in
parallel for a significant portion of the reduction.

A simple reduce, one made to create a final total from a mapping that produces
values of zero or one for all matching elements, can be processed as a less and
less parallel tree of reductions. In the first step, all reductions produce the
total of all odd-even pairs of elements, and produce a new list that goes
through the same process. This list reduction continues until there is only one
item left remaining. Of course this particular reduction is of very little use,
as each reduction is so trivial, you'd be better off assigning an Nth of the
workload to each of the N cores and doing one final summing. A more complex,
but equally useful reduction would be the concatenation of a chain of matrices.
Matrices are associative even if they are not commutative, and as such, the
chain can be reduced in parallel the same way building the total worked. By
maintaining the order during reduction you can apply parallel processing to
many things that would normally seem serial as long as they are associative in
the reduce step. Not only matrix concatenation, but also products of floating
point values such as colour modulation by multiple causes such as light,
diffuse, or gameplay related tinting. Building text strings can be associative,
as can be building lists and of course conc-trees themselves.
 
\section{Spatial sets for collisions}

In collision detection, there is often a broad-phase step which can massively
reduce the number of potential collisions we check against. When ray casting,
it's often useful to find the potential intersection via an octree, bsp, or
other single query accelerator. When running path finding, sometimes it's
useful to look up local nodes to help choose a starting node for your journey.

All spatial data-stores accelerate queries by letting them do less. They are
based on some spatial criteria and return a reduced set that is shorter and
thus less expensive to transform into new data.

Existing libraries that support spatial partitioning have to try to work with
arbitrary structures, but because all our data is already organised by table,
writing adaptors for any possible table layout is simple. Writing generic
algorithms becomes very easy without any of the side effects normally
associated with writing code that is used in multiple places. Using the table
based approach, because of its intention agnosticism (that is, the spatial
system has no idea it's being used on data that doesn't technically belong in
space), we can use spatial partitioning algorithms in unexpected places, such
as assigning audio channels by not only their distance from the listener, but
also their volume and importance. Making a 5 dimensional spatial partitioning
system, or an N dimensional one, would only have to be written once, have unit
tests written once, before it could be used and trusted to do some very strange
things. Spatially partitioning by the quest progress for tasks to do seems a
little overkill, but getting the set of all nearby interesting entities by
their location, threat, and reward, seems like something an AI might consider
useful.

\section{Lazy Evaluation for the masses}

When optimising Object\-oriented code, it's quite common to find local caches
of calculations done hidden in mutable member variables. One trick found in
most updating hierarchies is the dirty bit, the flag that says whether the
child or parent members of a tree imply that this object needs updating. When
traversing the hierarchy, this dirty bit causes branching based on data that
has only just loaded, usually meaning there is no chance to guess the outcome
and thus in most cases, causes a pipeline flush and an instruction lookup.

If your calculation is expensive, then you might not want to go the route that
renderer engines now use. In render engines, it's cheaper to do every scene
matrix concatenation every frame than it is only doing the ones necessary and
figuring out if they are.

For example, in the /emph{GCAP 2009 - Pitfalls of Object Oriented Programming
presentation by Tony Albrecht} in the early slides he declares that checking a
dirty flag is less useful than not checking it as if it does fail (the case
where the object is not dirty) the calculation that would have taken 12 cycles
is dwarfed by the cost of a branch misprediction (23-24 cycles).

If your calculation is expensive, you don't want to bog down the game with
a large number of checks to see if the value needs updating. This is the point
at which existence-based-processing comes into its own again as existence the
dirty table implies that it needs updating, and as a dirty element is
updated it can be pushing new dirty elements onto the end of the table, even
prefetching if it can improve bandwidth.

\section{Not getting what you didn't ask for}

When you normalise your data you reduce the chance of another multifaceted
problem of object-oriented development. C++'s implementation of objects forces
unrelated data to share cache-lines.

Objects collect their data by the class, but many objects, by design, contain
more than one role's worth of data. This is partially because object-oriented
development doesn't naturally allow for objects to be recomposed based on their
role in a transaction, and partially because C++ needed to provide a method by
which you could have object-oriented programming while keeping the system level
memory allocations overloadable in a simple way. Most classes contain more than
just the bare minimum, partially because of inheritance, and partially because
of the many contexts in which an object can play a part. Unless you have very
carefully laid out a class, many operations that require only a small amount of
information from the class, will load a lot of unnecessary data into the cache
in order to do so. Only using a very small amount of the loaded data is one of
the most common sins of the object-oriented programmer.

Every virtual call loads in the cache-line that contains the virtual-table
pointer of the instance. If the function doesn't use any of the class's early
data, then that will be cache\-line usage in the region of only 4\%. That's a
memory throughput waste, and cannot be recovered without rethinking how you
dispatch your functions. After the function has loaded, the program has to load
the data it wants to work on, which can be scattered across the memory
allocated for the class too. Sometimes you might organise the data so that the
function is accessing contiguous blocks of useful information, but someone
going in and adding new data at the top of the class can shift all the data
that was previously on one cache-line, into a position where it is split over
two, or worse, make the whole class unaligned causing virtually random timing
properties on every call. This could be even worse if what they have added is a
precondition that loads data from an unrelated area of memory, in which case it
would require that the load finished before it even got to the pipeline flush
if it failed the branch into loading the function body and the definitely
	necessary transformable data.

A general approach with the table formatted data is to preparse the table to
produce a job list. There would be one job list per transform type identified
by the data if it was to emulate a virtual call. Then, once built, transform
using these new job tables to drive the process. For each transform type , run
the transform over all the jobs in the job queue built for this transform. This
is much faster as the function used to transform the data is no longer coming
from a virtual lookup, but instead is implied by which table is being
processed, meaning no instruction cache misses after the first call to the
transform function, and not loading the transform if there are zero entries.

Another benefit is that the required data can be reorganised now that it's
obvious what data is necessary. This leads to simple to optimise data
structures compared to an opaque class that to some extent, pretends that the
underlying memory configuration is unimportant and encapsulated. This can help
with pre-fetching and write combining to give near optimal performance without
any low level coding.

\section{Varying Length Sets}

Throughout the techniques so far, there's been an implied table structure to
the data. Each row being a struct, or each table being a row of columns of
data, depending on the need of the transforms. When we normally do stream
processing, for example, with shaders, we normally use fixed size buffers. Most
work done with stream processing has this same limitation, we tend to have a
fixed number of elements for both sides. However, we saw that conc-trees solve
the problem of mapping from one size input to another size output, and that it
works by concatenating the output data into a cache-oblivious structure. This
structure is very general purpose and could be a basis for preliminary work
with map-reduce programming for your game, but there are cases where much
better structures exist.

For filtering, that is where the input is known to be superset of the output,
then there can be a strong case for an annealing structure. Like the
conc-trees, each transform thread has its own output, but instead of
concatenating, the reduce step would first generate a total and a start
position for each reduce entry and then process the list of reduces onto the
final contiguous memory.

If the filtering was a stage in a radix sort or something that uses a similar
histogram for generating offsets, then a parallel prefix sum would reduce the
time to generate the offsets. A prefix sum is the running total of a list of
values. The radix sort output histogram is a great example because the bucket
counts indicate the starting points through the sum of all histogram buckets
that come prior. $o_n = \sum_{i=0}^{n-1} b_i$. This is easy to generate in
serial form, but in parallel we have to consider the minimum required
operations to produce the final result. In this case we can remember that the
longest chain will be the value of the last offset, which is a sum of all the
elements. This is normally optimised by summing in a binary tree fashion.
Dividing and conquering: first summing all odd numbered slots with all even
numbered slots, then doing the same, but for only the outputs of the previous
stage.

\begin{displaymath}
	\xymatrix{
	A \ar[d] \ar[dr] & B \ar[d] & C \ar[d] \ar[dr] & D \ar[d] \\
	a \ar[d] & ab \ar[d] \ar[drr] & c \ar[d] & cd \ar[d] \\
	a & ab & abc & abcd }
\end{displaymath}

Then once you have the last element, backfill all the other elements you didn't
finish on your way to making the last element. When you come to write this in
code, you will find that these back filled values can be done in parallel while
making the longest chain. They have no dependency on the final value so can be
given over to another process, or managed by some clever use of SIMD.

\begin{displaymath}
	\xymatrix{
	a \ar[d] & ab \ar[d] \ar[dr] & c \ar[d] & abcd \ar[d] \\
	a & ab & abc & abcd }
\end{displaymath}

Also, for cases where the entity count can rise and fall, you need a way of
adding and deleting without causing any hiccups. For this, if you intend to
transform your data in place, you need to handle the case where one thread can
be reading and using the data that you're deleting. To do this in a system
where objects' existence was based on their memory being allocated, it would be
very hard to delete objects that were being referenced by other transforms. You
could use smart pointers, but in a multi-threaded environment, smart pointers
cost a mutex to be thread safe for every reference and dereference. This is a
high cost to pay, so how do we avoid it?

Don't ever delete.

Deletion is for wimps. If you are deleting in a system that is constantly
changing, then you would normally use pools anyway. By explicitly not deleting,
but doing something else instead, you change the way all code accesses data.
You change what the data represents. If you need an entity to exist, such as a
CarDriverAI, then it can stack up on your table of CarDriverAIs while it's in
use, but the moment it's not in use, it won't get deleted, but instead marked
as not used. This is not the same as deleting, because you're saying that the
entity is still valid, won't crash your transform, but can be skipped as if it
were not there until you get around to overwriting it with the latest request
for a CarDriverAI. Keeping dead entities around is as cheap as keeping pools
for your components.

\section{Bit twiddling decision tables}

Condition tables normally operate as arrays of condition flag bit fields. The
flags are the collected results of conditions on the data. But, if the bit
fields are organised by decision rather than by row, then you can call in only
the necessary conditions into different decision transforms.

If the bits are organised by condition, then you can run a short transform on
the whole collection of condition bits to create a new, simpler list of whether
or nots. For example, you may have conditions for decisions that you can map to
an alphabet of a-m, but only need some for making a decision. Imagine you need
to be sure that a,d,f,g are all false, but e,j,k need to all be true. Given
this logic, you can build a new condition, let's say q, that equals $( a \wedge
d \wedge f \wedge g ) \wedge \neg( e \vee j \vee k )$ which can then be used
immediately as a job list.

Organising the bits this way could be easier to parallelize as a transform that
produces a condition stream would be in contention with other processes if they
shared memory\footnote{This would be from false sharing, or even real sharing in
the case of different bits in the same byte}. The benefit to row based conditions
comes when the conditions change infrequently, and the number of things looking
at the conditions to make decisions is small enough that they all fit in a single
platform specific type, such as a 32bit or 64bit unsigned int. In that case, there
would be no benefit in reducing the original contention when generating the condition
bits, and because the number of views, or contexts about the conditions is low, then
there is little to no benefit from splitting the processing so much.

If you have an entity that needs to reload when their ammo drops to zero,
and they need to consider reloading their weapon if there is a lull in the
action, then, even though both those conditions are based on ammo, the decision
transforms are not based on the same condition. If the condition table has been
generated as arrays of condition bitfields with each bit representing one row's
state with respect to that condition, then you can halve the bandwidth to the
check for definite-reload transform, and halve the bandwidth for the
reload-consideration check. There will be one stream of bits for the condition
of ammo equal to zero, and another stream for ammo less than max. There's no
reason to read both, so we don't.

What we have here is another case of deciding whether to go with structures of
arrays, or sticking with an array of structures.  If we access the data from a
few different contexts, then a structure of arrays trumps it. If we only have
one context for using the conditions, then the array of structures, or in this
case, array of masks, wins out. But remember to profile as any advice is only
advice and only measuring can really provide proof, or evidence that
assumptions are wrong.

\section{Joins as intersections}

Sometimes, normalisation can mean you need to join tables together to create
the right situation for a query. Unlike RDBMS queries, we can organise our
queries much more carefully and use the algorithm from merge sort to help us
zip together two tables. As an alternative, we don't have to output to a table,
it could be a pass through transform that takes more than one table and
generates a new stream into another transform. For example, per
entityRenderable, join with entityPosition by entityID, to transform with
AddRenderCall( Renderable, Position )

\section{Data driven techniques}

Apart from finite state machines there are some other common forms of data
driven coding practices, some of which are not very obvious, such as callbacks,
and some of which are very much so, such as scripting. In both these cases,
data causing the flow of code to change will cause the same kind of cache and
pipe-line problems as seen in virtual calls and finite state machines.

Callbacks can be made safer by using triggers from event subscription tables.
Rather than have a callback that fires off when a job is done, have an event
table for done jobs so that callbacks can be called once the whole run is
finished. For example, if a scoring system has a callback from "badGuyDies",
then in an Object-oriented message watcher you would have the scorer increment
its internal score whenever it received the message that a badGuyDies. Instead
run each of the callbacks in the callback table once the whole set of badguys
has been checked for death. If you do that, and execute every time all the
badGuys have had their tick, then you can add points once for all badGuys
killed. That means one read for the internal state, and one write. Much better
than multiple reads and writes accumulating a final score.

For scripting, if you have scripts that run over multiple entities, consider
how the graphics kernels operate with branches, sometimes using predication and
doing both sides of a branch before selecting a solution. This would allow you
to reduce the number of branches caused merely by interpreting the script on
demand.  If you go one step further an actually build SIMD into the scripting
core, then you might find that you can run script for a very large number of
entities compared to traditional per entity serial scripting. If your SIMD
operations operate over the whole collection of entities, then you will pay
almost no price for script interpretation\footnote{Take a look at the section
headed \emph{The Massively Vectorized Virtual Machine} on the bitsquid blog
http://bitsquid.blogspot.co.uk/2012/10/a-data-oriented-data-driven-system-for.html}.
