\chapter{Concurrency}

Any book on games development practices for contemporary and future hardware
must cover the issues of concurrency. There will come a time when we have more
cores in our computers than we have pixels on screen, and when that happens, it
would be best if we were already coding for it, coding in a style that allows
for maximal throughput with the smallest latency. Thinking about how to solve
problems for five, ten or even one hundred cores isn't going to keep you safe.
You must think about how your algorithms would work when you have an infinite
number of cores. Can you make your algorithms work for N cores?

Writing concurrent software has been seen as a hard task in the past because
most people think they understand threading and can't get their heads around
all the different corner cases that are introduced when you share the same
memory as another thread. Fixing these with mutexs and critical sections can
become a minefield of badly written code that works only 99\% of the time. For
any real concurrent development we have to start thinking about our code
transforming data. Every time you get a deadlock or a race condition in
threaded code, it's because there's some ownership issue. If you code from a
data transform point of view, then there are some simple ground rules that
provide very stable tools for developing truly concurrent software.

\section[Thread-safe]{What it means to be thread-safe}

Academics consistently focus on what is possible and correct, rather than what
is practical and usable, which is why we've been inundated with multi-threaded
techniques that work, but cause a lot of unnecessary pain when used in a high
performance system such as a game. The idea that something is thread-safe
implies more than just that it is safe to use in a multi-threaded environment.
There are lots of thread safe functions that aren't mentioned becuase they seem
trivial, but it's a useful distinction to make when you are tracking down what
could be causing a strange thread issue. There are functions without
side-effects, such as the intrinsics for sin, sqrt, that return a value given a
value. There is no way they can cause any other code to change behaviour, and
no other code can change its behaviour either\footnote{If you discount the
possibility of other code changing the floating point operation mode}. In
addition to these very simple functions, there are the simple functions that
change things in an idempotent fashion, such as memset.

Thread safe implies that it doesn't just access its own data, but accesses some
shared data without causing the system to enter into an inconsistent state.
Inconsistent state is a natural side effect of multiple processes accessing and
writing to shared memory. It is these side effects that are the cause of many
bugs in multi-threaded code, which is why the develpers of the Erlang language
chose to limit the programmer to code that doesn't have side-effects.  Any code
that relies on reading from a shared memory before writing back an adjusted
value can cause inconsistent state as there is no way to guarantee that the
writing will take place before anyone else reads it before they modify it.

\begin{verbatim}
int shared = 0;
void foo() {
	int a = shared;
	a += RunSomeCalculation();
	shared = a;
}
\end{verbatim}

Making this work in practice is hard and expensive. The standard technique used
to ensure the state is consistent is to make the value update atomic. How this
is achieved depends on the hardware and the compiler. Most hardware has an
atomic instruction that can be used to create thread-safety through mutual
exclusions. On most hardware the atomic instruction is a compare and swap, or
CAS. Building larger tools for thread-safety from this has been the mainstay of
multi-threaded programmers and operating system developers for decades, but
with the advent of multi-core consoles, programmers not well versed in the
potential pitfalls of multi-threaded development are suffering because of the
learning cliff involved in making all their code work perfectly over six or
more hardware threads.

Using mutual exclusions, it's possible to rewrite the previous example:

\begin{verbatim}
int shared = 0;
Mutex sharedMutex;
void foo() {
	sharedMutex.acquire();
	int a = shared;
	a += RunSomeCalculation();
	shared = a;
	sharedMutex.release();
}
\end{verbatim}

And now it works. No matter what, this function will now always finish its
task without some other process damaging its data. Every time one of the
hardware threads encounters this code, it stops all processing until the mutex
is acquired. Once it's acquired, no other hardware thread can enter into these
instructions until the current thread releases the mutex at the far end.

Every time a thread-safe function uses a mutex section, the whole machine stops
to do just one thing. Every time you do stuff inside a mutex, you make the code
thread-safe by making it serial. Every time you use a mutex, you make your code
run bad on infinite core machines.

Thread-safe, therefore, is another way of saying: not concurrent, but won't
break anything.  Concurrency is when multiple threads are doing their thing
without any mutex calls, semaphores, or other form of serialisation of task.
Concurrent means at the same time. A lot of the problems that are solved by
academics using thread-safety to develop their multi-threaded applications
needn't be mutex bound. There are many ways to skin a cat, and many ways to
avoid a mutex. Mutex are only necessary when more than one thread shares write
privileges on a piece of memory. If you can redesign your algorithms so they
only ever require one thread to be given write privilege, then you can work
towards a fully concurrent system.

Ownership is key to developing most concurrent algorithms. Concurrency only
happens when the code cannot be in a bad state, not because it checks before
doing work, but because the design is such that no process can interfere with
another in any way.

\section{Inherently concurrent operations}

When working with tables of data, many operations are inherently concurrent.
Simple transforms that take one table and generate the next step, such as those
of physics systems or AI state / finite state machines, are inherently
concurrent. You could provide a core per row / element and there would be no
issues at all. Setting up the local bone transforms from a skeletal animation
data stream, ticking timers, producing condition values for later use in
condition tables. All these are completely concurrent tasks. Anything that
could be implemented as a pixel or vertex shader is inherently concurrent,
which is why parallel processing languages such as shader models, do not
cheaply allow for random write to memory, and don't allow accumulators across
elements.

Seeing that these operations are inherently concurrent, we can start to see
that it's possible to restructure our game from an end result perspective. We
can use the idea of a structured query to help us find our critical path back
to the game state. Many table transforms can be split up into much smaller
pieces, possibly thinking along the lines of map reduce, bringing some
previously serial operations into the concurrent solution set.

Any N to N transform is perfectly concurrent. Any N to <=N is perfectly
concurrent, but depending on how you handle output NULLs, you could end up
wasting memory. A reduce stage is necessary, but that could be managed by a
gathering task after the main task has finished, or at least, after the first
results have started coming in.

Any uncoupled transforms can be run concurrently. Ticking all the finite state
machines can happen at the same time as updating the phsyics model, can happen
at the same time as the graphics culling system building the next frame's
render list. As long as all your different game state transforms are
independent from each other's current output, they can be dependent on each
other's original state and still maintain concurrency.

For example, the physics system can update while the renderer and the AI rely
on the positions and velocities of the current frame. The AI can update while
the animation system can rely on the previous set of states.

Multi stage transforms, such as physics engine broadphase, detection, reaction
and resolution, will traditionally be run in series while the transforms inside
each stage run concurrently. Standard solutions to these stages require that
all data be finished processing from each previous stage, but if you can find
some splitting planes for the elements during the first stage, you can then
remove temporal cohesion from the processing because you can know what is
necessary for the next step and hand out jobs from finished subsets of each
stage's results.

Concurrent operation assumes that each core operating on the data is free to
access that data without interfering, but there is a way that seemingly
unconnected processes can end up getting in each other's way. Most systems have
multiple layers of cache, and this is where the accident can happen. False
sharing is when data, though actually unrelated, is connected by the physical
layout of the hardware. For example, the cachelines of a CPU might be 16 to 128
bytes long. If two different CPUs try to write to neighbouring bytes, or words,
at best the cores will lock up as the memory is shunted around trying to keep
memory consistent, but worse, could end up losing data if the cache is not
coherent\footnote{cache coherency is the system by which changed data is
propagated to other caches so that consistent state is achieved.}. To reduce
the chance of this happening, processing of small element tables should
consider this and split any cooperation into cacheline size jobs.

\section[Gateways]{Queues as gateways, and "Now"}

When you don't know how many items you are going to get out of a transform,
such as when you filter a table to find only the X that are Y, you need run a
reduce on the output to make the table non-sparse. Doing this can be log(N)
latent, that is pairing up rows takes serial time based on log(N). But, if you
are generating more data than your input data, then you need to handle it very
differently. Mapping a table out onto a larger set can be managed by generating
into gateways, which once the whole map operation is over, can provide input to
the reduce stage. The gateways are queues that are only ever written to by the
Mapping functions, and only ever read by the gathering tasks such as Reduce.
there is always one gateway per Map runtime. That way, there can be no sharing
across threads. A Map can write that it has put up more data, and can set the
content of that data, but it cannot delete it or mark any written data as
having been read. The gateway manages a read head, that can be compared with
the write head to find out if there is any waiting elements. Given this, a
gathering gateway or reduce gateway can be made by cycling through all known
gateways and popping any data from the gateway read heads. This is a fully
concurrent technique and would be just as at home in variable CPU timing
solutions as it is in standard programming practices as it implies a consistent
state through ownership of their respective parts. A write will stall until the
read has allowed space in the queue. A read will either return "no data" or
stall until the write head shows that there is something more to read.

When it comes to combining all the data into a final output, sometimes it's not
worth recombining it into a different shape, in which case we can use
conc-trees, a technique that allows for concurrency and cache-friendly
transforming while maintaining most of the efficiency of contiguous arrays.
Conc-trees are a tree representation of data that almost singlehandedly allows
for cache oblivious efficient storage of arbitrary lists of data. Cache
oblivious algorithms don't need to know the size of the cache in order to fully
utilise them and normally consist of some kind of divide and conquer core
algorithm, and conc-trees do this by either representing the null set, a
concatenation, or some data. If we use conc trees with output from transforms,
we can optimise the data that is being concatenated so it fits well in cache
lines, and we can also do the same with conc tree concatenate nodes, making
them concatenate more than just two nodes, thus saving space and memory access.
We can tune these structures per platform or by data type.

Concurrency is acheived between writer and readers by preparing a new tree from
the remnants of the old tree before pushing the root node back. Normally,
operations involve adding or subtracting elements from a list: simply
recreating the small array that contains the change, then recreating all the
nodes all the way back to the root. This allows for a new tree to exist
alongside the old tree. Maintaining reference counts for reading will allow for
perfect timed destruction of the old nodes, but you may find there will be a
time when you know there won't be any remaining readers, leading to a simpler
garbage collection. Multiple writers will still block, but readers can operate
off old data while a writer is recreating.

Given that we have a friendly structure for storing data, and that these can be
built as concatenations rather than some data soup, we have basis for a nicely
concurrent yet associative transform system. If we don't need to keep the
associative nature of the transform, then we can optimise further, but as it
comes at little to no cost, and most reduce operations rely on associativity,
then it's good that we have that option.

Moving away from transforms, there is the issue of \emph{now} that crops up whenever
we talk about concurrent hardware. Sometimes, when you are continually updating
data in real time, not per frame, but actual real time, there is no safe time to
get the data, or process it. Truly concurrent data analysis has to handle
reading data that has literally only just arrived, or is in the process of
being retired. data-oriented development helps us find a solution to this by
not having a concept of now but a concept only of the data. If you are writing
a network game, and you have a lot of messages coming in about a player, and
their killers and victims, then to get accurate information about their state,
you have to wait until they are already dead. With a data-oriented approach,
you only try to generate the information that is needed, when it's needed. This
saves trying to analyse packets to build some predicted state when the player
is definitely not interesting, and gives more up to date information as it
doesn't have to wait until the object representing the data has been updated
before any of the most recent data can be seen or acted on.

The idea of now is present only in systems that are serial. There are multiple
program counters when you have multiple cores, and when you have thousands of
cores, there are thousands of different nows. Humans think of things happening
at a certain time, but sometimes you can take so long doing something that more
data has arrived by the time you're finished that you should probably go back
and try again.

Take for example the idea of a game that tries to get the lowest possible
latency between player control pad and avatar reaction. In a lot of games, you
have to put up with the code reading the pad state, the pad state being used to
adjust animations, the animations adjusting the renderables, the rendering
system rastering and the raster system swapping buffers. In most games it takes
at least three frames

\begin{quote}
{\bf read} the player pad. \newline
{\bf respond} in logic to new pad state. \newline
{\bf animate} the character. \newline
{\bf queueForRender} the new bone positions. \newline
{\bf FRAME} happens, moves the render requests into the procesing list while calling: \newline
{\bf renderToBackBuffer} on the old lists. \newline
{\bf FRAME} happens again which starts to render our reaction to the pad state. \newline
{\bf swapBuffers} then shows our change as \newline
{\bf VISIBLE}
\end{quote}

and many games have triple buffering and animation systems that don't update
instantly.

In this case, the player could potentially be left until the last minute before
checking pad status, and apply an emergency patchup to the in flight
renderQueue. If you allowed a pad read to adjust the animation system's history
rather than it's current state, you could request that it update it's
historical commits to the render system, and potentially affect the next frame
rather than the frame three buffer swaps from now. Alternatively, have some of
the game run all potential player initiated events in parallel, then choose
from the outcomes based on what really happened, then you can have the same
response time but with less patching of data.


