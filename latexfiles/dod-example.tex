\chapter{In Practice}\label{chap:inpractice}

Data-oriented development is not rooted in theory, but practice. Because of
this, it's hard to describe the methodology without some practical examples. In
this chapter I will document some experiences with the data-oriented approach.

\section{Data-manipulation}

\subsection{The Cube}

In 22cans first experiment, there was call to handle a large amount of traffic
from a large number of clients, potentially completely fragmented, and yet also
completely syncronised. The plan was to develop a service capable of handling a
large number of incoming packets of \emph{update} data while also compiling it
into compressed \emph{output} data that could be send to the client all with a
minimal turnaround time.

The system was developed in C++11 on Linux, and followed the basic tenets of
data-oriented design from day one. The data was loaded, but not given true
context.  Manipulations on the data were procedures that operated on the data
given other data. This was a massive time saver when the servers needed a
complete redesign. The speed of data processing was sufficient to allow us to
run all the services in debug\footnote{anyone remembering developing on a PS2
will likely attest to the minimal benefit you get from optimisations when your
main bottleneck is the VU and GS. The same was true here, we had a simple
bottleneck of the size of the data and the operations to run on it.
Optimisations had minimal impact on processing speed, but they did impact our
ability to debug the service when it did crash.}.

When the server went live, it wasn't the services that died, it was the front
end. Nginx is amazing, but under that amount of load on a single server, with
so many of the requests requiring a lock on an SQL db backend, the machine
reached it's limit very quickly. For once, we think PHP itself wasn't to blame.
We had to redesign all the services so they could work in three different
situations so as to allow the server to become a distributed service. By not
locking down data into contexts, it was relatively easy to change the way the
data was processed, to reconfigure the single service that previously did all
the data consumption, collation, and serving, into three different services
that handled incoming data, merging the multiple instances, and serving the
data on the instances. In part this was due to a very procedural approach, but
it was also down to not linking together data that was separate by binding into
an object context. The lack of binding allows for simpler recombination of
procedures, simpler rewriting of procedures, and simpler repurposing of
procedures from related services. This is something that object oriented
approach makes harder because you can be easily tempted to start adding base
classes and inheriting to gain common functionality or algorithms. As soon as
you do that you start tying things together by what they mean rather than what
they are, and then you lose the ability to reuse code.

\subsection{Rendering order}

While working on the in-house engine at Broadsword, we came up with an idea for
how to reimplement the renderer that should have been much more efficient, not
just saving CPU cycles, but also allowing for platform specific optimisations
at run time by having the renderer analyse the current set of renderables and
organise the whole list of jobs by what caused the least program changes,
texture changes, constant changes and primitive render calls. The system seemed
too good to be true, and though we tried to write the system, Broadsword never
finished it. After Broadsword dissolved, when I did finish it, I didn't have
access to a console\footnote{The target platforms of the engine included the
Playstation 2 and the Nintendo Wii along with Win32.}, so I was only able to
test out the performance on a PC.  As expected the new system was faster, but
only marginally. With hindsight, I now see that the engine was only marginally
more efficient because the tests were being run on a system that would only see
marginal improvements, but even the x86 architecture saw improvements, which
can be explained away as slightly better control flow and generally better
cache utilisation.

First let me explain the old system.

All renderables came from a scene graph. There could be multiple scene graph
renders per frame, which is how the 2D and 3D layers of the game were
composited, and each of them could use any viewport, but most of them just used
the fullscreen viewport as the engine hadn't been used for a split screen game,
and thus the code was probably not working for viewports anyway. Each scene
graph render to viewport would walk the scene\footnote{sometimes multiple
cameras belonged to the same scene to the scene graph would be walked multiple
times} collecting transforms, materials, colour tints, and meshes, to render,
at which point the node that contained the mesh would push a render element
onto the queue for rendering into that viewport. This queueing up of elements
to render seemed simple and elegant at the time. It meant that programmers
could quickly build up a scene and render it, most of the time using helpers
that loaded up the assets and generated the right nodes for setting textures,
transforms, shader constants, and meshes. These rendering queues were then
sorted before rendering. Sorted by material only for solid textures, and sorted
back to front for alpha blended materials. This was the old days of fixed
function pipelines and only minimal shader support on our target platforms.
Once the rendering was done, all the calculated combinations were thrown away.
This meant that for everything that was rendered, there was definitely a
complete walk of the scene graph.

The new system, which was born before we were aware of data-oriented design,
but was definitely born of looking at the data, was different in that it no
longer required walking the scene graph. We wanted to maintain the same
programmer friendly front edge API, so maintained the facade of a scene graph
walk, but instead of walking the graph, we only added a new element to the
rendering when the node was added, and only removed it when it was removed from
the scene graph. This meant we had a lot of special code that looked for
multiple elements registered in multiple view port lists, but, other than that,
a render merely looked up into the particular node it cared about, gathered the
latest data, and processed it pulling when it required it rather than being
pushed things it didn't even care about.

The new system benefited from being a simple list of pointers from which to
fetch data (the concept of a dirty transform was removed, all transforms were
considered to be dirty every frame), and the computation was simplified for
sorting as all the elements that were solid were already sorted from the
previous render, and all the alpha blended elements were sorted because they
belonged to the set of alpha blended elements. This lack of options accounted
for some saved time, but the biggest saving probably came from the way the data
was being gathered per frame rather than being generated from a incoherent
tree. A tree that, to traverse, requried many pointer lookups into virtual
tables as all the nodes were base classed to a base node type and all update
and render calls were virtual causing many misses all the way through each of
the tree walks.

In the end the engine was completely abandoned as Broadsword dissolved, but
that was the first time we had taken an existing codebase and (though
inadvertantly) converted it to data-oriented.

\subsection{Keeping track of damage}

During the development of the multiplayer code for Max Payne 3 there came a
point where it became more and more important that we kept track of damage
dealt to every player from every player. Sometimes this would be because we
want to estimate more accurately how much health a player had. Sometimes this
would be because we needed to know who saved who and thus who got a teammate
saved bonus, or who got the assist. We kept a tight ship on Max Payne 3, we
tried to keep the awards precise as far as we could go, no overcompensating for
potential lost packets, no faking because we wanted a fair and competition
grade experience where players could make all the difference through skill, and
not through luck. To do this, we needed to have a system that kept track of all
the bullets fired, but not fail when we missed some, or they arrived out of
order. What we needed was a system that could be interrogated, but could
receive data about things that had happened back in time. It needed to be able
to heal itself in case of oddities, and last but not least, it needed to be
really quick so we can interrogate it many times per frame.

The solution was a simple list of things that happened and at what time.
Initially the data was organised as a list of structures with a sorted list of
pointers to the data for quick queries about events over time, but after
profiling the queries with and without the sorted list, the benefits of the
lack of maintaining a sorted list outweighed the benefits of doing a one time
sort on just the data needed to satisfy the query. There were only two queries,
which were not frequently called, that benefitted from the sorted data, and
when they were called they didn't need to be extremely fast as they were called
as the result of player death, which not only happened less often than once a
frame, but also regularly resulted in there being less to update as the player
cared a lot less about visibility checks and handling network
traffic\footnote{dead men don't care about bullets and don't really care that
much about what other players are doing}. This simple design allowed for many
different queries to run over the core data, and allowed for anyone to add
another new query easily because there was no data-hiding getting in the way of
any new kind of access. Object-oriented approaches to this kind of data
handling often provide a gateway to the data but marshall it so that the
queries become heavyweight and consistently targetted for optimisation. With a
very simple data structure and open access to the data in any form, any new
query could be as easily optimised as any existing one.

\section{Game entities}

\subsection{Converting an object oriented player}

While I was at Frontier Developments, I was working on a little space game that
ended up being a boat game in caverns\footnote{due to concerns that a 2D space
game might be misinterpretted as the next elite I imagine} and that was the
first time I intentionally changed a game from object-oriented to
component-oriented. The ships all had their rendering positions, their health,
speed, momentum, ammo recharge timers etc, and these were all part of a base
ship class that was extended to be a player class, and inherited from a basic
element that was extended to also include the loot drops and the dangerous ice
blocks. All this is pretty standard practice, and from the number of codebases
I have seen, it's pretty light on the scale of inheritance with which games
normally end up. Still, I had this new tool in my bag, the component-oriented
approach specifically existence based processing. I wanted to try it out, see
if it really did impact the profile of the game in any significant way. I
started with the health values. I made a new component for health, and anyone
that was at full health, or was dead, was not given an instance of the
component. The player ship and the enemy ships all had no health values until
they were shot and found wanting.

Transitioning to the component based health took a while. Translating any game
from one programming paradigm to another is not a task to be taken lightly,
even when your game code is small and you've only spent sixty hours developing
the game so far. Once the game was back up though, the profile spoke loud and
clear of the benefits. I had previously spent some time in health code every
frame, this was because the ships had health regenerators that ran and updated
the health every time they got an update poll. They needed to do an update
because they were updated, not because they needed to, and they didn't have any
way to opt out before the componentisation, as the update to health was part of
the ship update. Now the health update only ran over the active health
instances, and removed itself once it reached max health rather than bounce of
the max health.

This relatively small change allowed me to massively increase the number of
small delicate ships in the game (ships that would beat the player ship by
overrunning it rather than being able to withstand the player's fire and get
close in to do some damage), and also lead to another optimisation:
componentising the weapon recharge timer.

If the first change was a success, then the second was a major success.
Instead of keeping track of when a weapon was ready to fire again, the time
that it would be available was inserted into a sorted list of recharge times.
Only the head of the list was checked every global update, meaning that a large
number of weapons could be recharging at once and none of them caused any data
access until they were very nearly or actually ready.

This immediate success lead me to believe the data-oriented design movement was
really important and needed to be spread around, and probably caused my sudden
hatred of object-oriented programming. From that point on, all I could see was
cache-misses and pointless update checks.

\subsection{People don't really exist.}

\begin{quote}
\em \small
A Mathematician, a Biologist and a Physicist are sitting in a street cafe 
watching people going in and coming out of the house on the other side of 
the street.

First they see two people going into the house. Time passes.  After a while 
they notice three persons coming out of the house.

The Physicist: "The measurement wasn't accurate.". \\
The Biologists conclusion: "They have reproduced". \\
The Mathematician: "If now exactly 1 person enters the house then it will 
be empty again."
\end{quote}

The followers in the GODUS prototype are little objects when they are running
around, and though the code has changed quite a bit from the initial lists of
pointers to people structures, and has thus become unweildy, there was one
element that was a perfect example of hierarchical level of detail in the
game logic. The followers, once they had started to build a building, dissappeared.
They were no longer required in any way, so they were reduced to a mere
increment of the number of people in a building. An object-oriented
programmer disputed this being a good way to go, "but what if the person has a
special weapon, or is a special character?". Logic prevailed. If there was
a special weapon, then there would have been a count of those in the house, or
the weapon would have become owned by the house, and as for a special
character, the same kind of exception could be made, but in reality, which is
where we firmly plant ourselves when developing in the data-oriented paradigm,
these potential changes had not so far been requested, and by the end of the
development, they still hadn't.

Another example from the GODUS prototype was the use of duck-typing. Instead of
adding a base class for people and houses, just to see if they were meant to be
under control of the local player, we used a templated function to extract a
boolean value. Duck typing doesn't require anything more than certain member
functions or variables being available. They don't have to be in the same place, or come
bundled into a base class: they just have to be present.

When teaching data-oriented design, I find the biggest hurdle is convincing
programmers that data-oriented design is possible for some areas of
development.  It's very easy to accidentally assume that you need to bundle
things into objects, especially after years of training and teaching
object-oriented developlment. It won't come naturally and you will keep catching
yourself building things as objects first then making them more relational
afterwards.  It will take time to fully remove the muscle memory of putting
related things in the same class, but worry not, I'm sure all data-oriented
developers go through this transition where they know they're not coding
data-oriented, but they don't quite know how to do it yet.

\subsection{Lazy evaluation molasses}

The idea of lazy evaluation makes so much sense, and yet it's precisely that
kind of sense that makes no sense for a computer. I remember there was a dirty
bit check before an update in some of the global update code on Outsider,
something I would have done a hundred times over before I started to get a
better feel for what the computer was doing, and found one of the best
lines-of-code to time-saved ratio fixes I ever found in a triple A game.  The
dirty bit check was just before the code that actually did the work of updating
the instance. This meant that the CPU had to preload the \emph{fixup} code while
it was checking the dirty bit, and because the fixup code was virtual functions
based, it meant that the code was loading the virtual table value and the
function was preloaded all the while the dirty bit was about to say it wasn't
necessary to do any of it. To make matters worse, the code was inside an if,
implying to most compilers that it would be the more likely taken
branch\footnote{profile guided optimisation might have saved a lot of time
here, but the best solution is to give the profiler and optimiser a better
chance by leaving them less to do}. The fix was simple. Build up a local array
of objects to actually do an update of, then do them all at once. Because the
code to do the update wasn't preloaded, the whole update took a lot less time.
You can have your lazy evaluation, but don't preload the evaluation if you
don't have to.

\subsection{Component oriented design}

While at Broadsword, I found an article about dungeon seige proclaiming the
benefit of components when developing a game. The GOs or Game Objects that were
used were components that could be stitched together to create new compound
objects.

Taking that as inspiration, I looked at all our different scene-graph nodes,
gameplay helper classes and functions we commonly used, and tried to distill
them into their elements. I started by just making components such as
\emph{PlayerCharacter} and \emph{AICharacter}, adding a \emph{Prop} element
before realising where I was going wrong. The object oriented mindset had told
me to look at the compounds, not the elements, so I went back to the drawing
board and started dissecting the objects again. Component oriented development
works out best when you completely explode all your compounds, then recombine
when you find the combinations are consistent, or when combining makes more
sense to computation.

After I was done fully dissecting our basic game classes I had a long list of
separate, elements that by themselves did very little\footnote{ Position,
Velocity, PlayerInputMapper, VelocityFromInput, MeshRender, SkinnedMeshRender,
MatrixPalette, AnimPlayer, AnimFromInput, and CollisionHandler}. At this point,
I added the bootstrap code to generate the scene. Some components required
other components at runtime, so I added a \emph{requires} trait to the
components, much like \#include, except to make this work for teardown, I also
added a count so components went away much like reference counted smart
pointers folded away. The initial demo was a simple animated character running
around a 3D environment, colliding with props via the collision Handler.

In the beginning, I had a container entity that maintained a list of
components.  Once I had a first working version, I stepped back and thought
about the system. I started to see that the core entity wasn't really
necessary. As long as any update components were updated, then the game would
tick on. As long as the entity's components could find their requirement
components, then the game would work. I shifted to having an implicit entity
based on the UID I generate on entity creation. This meant that all entities
were really only the components that were linked to that ID, as the ID didn't
index into an array of entity objects, or point to an allocated entity, it was
merely an ID off which to hang all the components.

Adding features to a class at runtime was now possible. I could inject an
additional property into the existing entities because I had centred the
entities around a key, and not any kind of central class. The next step from
here was to add components via a scripting language so it would be possible to
develop new gameplay code while running the game. Unfortunately this never
happened, but some MMO engine developers have done precisely this.

The success of this demo convinced me that components would play a major role
in our continuing game development, but alas, we only used the engine for one
more product and the time to bring the new component based system up to speed
for the size of the last project was estimated to be counter productive. Mick
West released an article about how when he was at Neversoft, he did manage to
convert the Tony Hawks code-base to
components\footnote{http://cowboyprogramming.com/2007/01/05/evolve-your-heirachy/}
which means it's not impossible to migrate, it's just not easy. If we were to
start component oriented... well that's a different story, normally told by
Scott Bilas.
