\chapter{Counting and Domain knowledge}

Counting is something we take for granted. We count things and ignore things
based on probabilities instantly calculated from the numbers obvious to us.
Counting is a useful tool when it comes to processing information, saving space
while doing transformations, and for writing a problem in a more mathematically
correct manner, which usually leads to a more efficient solution and one that
can be more easily generalised or reused. To some extent, we also count on
things that can't be counted. We consider the chance of something happening,
and when it's going to happen and take that into account. This is something
that we have no way of teaching compilers as it is different every time and is
often part of the design. We must assume they cannot be taught this until the
compilers become the programmers. This information is what we call domain
knowledge.

Some languages can be seen to count, functional languages to a certain extent
always count, but C++ does not. Most imperative languages don't count other
than in the optimisation stage, the stage when loops are unrolled, branches
that are controlled by constants are reduced, and constant expressions are
brought to their minimal representation. All this helps reduce the size of a
program, and also helps increase its speed, but the level of counting available
to a compiler is quite simple compared to what humans can accomplish given the
full description of a problem to solve. The problem that a piece of software is
designed to solve is not transparent to the compiler, the assembler, or the CPU
that runs the machine code. The computer, the language, nor programming
paradigm will solve the problem of transferring the entire design and all the
domain knowledge into the executable code, whether static, or dynamic such as
in the case of runtime compiling languages or interpreters that can optimise
the bytecode at runtime.

Counting is not simple. Counting is an area of Mathematics linked with
probability and very large numbers, and in such is the cornerstone of most
forms of cryptography.  Handling very large numbers, but accurately, is not
something computers are very good at. Computers do well with very large amounts
of medium sized numbers. A computer is very happy working on millions of
different numbers ranging in the millions, and 64 bit computers are happy all
the way up to large quadrillions, but when the size of numbers is measured in
thousands of zeros, computers have to give up and resort to just doing what
they're told.

The simple factorial is a function that usually gets out of hand for values
greater than 20 ($20! = 2432902008176640000$ is the
largest factorial less than $2^{64} = 18446744073709551616$) and when we're
working with taking items from a list of possible items, the factorial function
becomes a common tool.

When we think about a problem, we often have an idea of how many things we are
working with. A compiler cannot guess at these things as it has to work at
runtime with only the information given at compile time. Branch predictors
help with some aspects of this lack of runtime rebuilding, as they allow a
program to respond to actual counts by changing the default flow. We cannot
hope that our statically compiled programs can count like the runtime compiled
languages or the bytecode languages that have runtime optimisers.

And so we have now a vague idea of what counting means. We see how humans are
currently the only means by which a computer can count on any large scale, as
only humans can grasp the concept for which the programs are designed.
Counting means to understand the larger picture of what kinds of numbers of
events and the expected type and value of the elements that are to be
processed. This forward thinking can be intuitive reasoning based on the design
of the software (in a game, an entity may only be able to die once) or may be
garnered by experimentation (conducting tests is a very good way to begin any
refactoring) or third part produced evidence (if an area of development is new
to you, you should see if someone else has publicly published any papers or
reports,) but will not likely ever be part of an optimising compiler, so we
must learn to count for our programs.

\section{Learning to count}

Sometimes we just want to save space, and to do that we need to understand the
data being compressed. The compiler cannot do it for us, and we will always
have a better idea of what the data is meant to be like. The best example of
the difference between trying to compress anything and being able to specialise
must be the special case lossy compressors: JPEG, MPG, and the like. In these
cases, and others where the information is held inside the data, but not all
the data is required to rebuild the information, only an external agent that is
aware of what actually constitutes important data can provide the information
about what can be ignored. Lossy algorithms are based on what can be expected
(predictions which might require a transform into a different way of viewing
the data), and what is fundamental (some basic structure that is known, but
would not be picked up by a standard compression technique). All lossy
techniques rely on knowing the difference between what is close enough and what
is unacceptable degradation. Lossy techniques rely on biological imperfections,
something which is still hard to build a general optimiser for.

Lossless compression relies on being able to count, and in one way, we have
invented counting algorithms in forms of Huffman or arithmetic encoding. These
algorithms count before they compress, to determine the chance of any
particular element being seen. Many lossless compressors take an approach like
this, or use history relative probability, such as run-length-encoding where
repeating a value is one of two high probability choices available during
encoding. Not being able to look at the data beforehand and come up with a
realistic, or even reasonable estimate of probability of each possible
encodable value will hamper any form of compression that relies on
probabilities.

Without understanding the cause of the data, these algorithmic approaches can
only go so far. If we know the reason why the data exists, as in where it comes
from or what it's for, there are often opportunities for even better
compression. For example, compressing the RGB in an image with alpha, knowing
that the data is bitwise AND onto destination data that has a known set of bits
set, knowing the ranges of many parts of a document and knowing biases for
values given other values in the same document.


%basic compression of some data
%add constraints, watch size go down
%preshing fit 1,000,000 ints into 1MB

\section{Counting to find out facts}

Counting isn't just about compression, sometimes it's about speed. Given an
algorithm, sometimes the best way to make it faster is to find out what data
it's being given in the first place and find out if it comes in regular
patterns, or is large enough or small enough to change the algorithm. In
data-oriented design, it's always pertinent to perform some analysis of the
real world performance of your code, and sometimes what you count can be
very important.

If a function being called by iterating over a large set of data and being
called on every element in the list, then the function can be timed in various
ways. You can time how long it takes to do the whole set of updates, then
divide by the number of updates, or you can try to profile the function per
call. You're probably going to get reasonably accurate timings either way, but
what if the function was only going to be called if some value in the element
was set? What if the function is normally fast but has spikes?

In the case of spikes, this is where per call timings are essential. Counting
the nanoseconds for operations (and storing the arguments going into them) is a
great way to do analysis of why your function has unexpected performance.

In some cases, checking whether or not to do work just before doing it can
bring about significant performance issues. If the function to run is not known
at compile time, then there is a higher chance that this pattern exhibits bad
performance. Take the dirty flag as an example. If an entity has a function
that can be made dirty and only needs to be updated when that flag is dirty,
then it makes sense to check that flag before committing to doing any work, but
sometimes you can't stop the compiler from assuming the probability of that is
high enough to begin doing the work before you're sure it's necessary.

%dirty flag code example

This code runs quickly and efficiently while the dirty flag is true, but the
moment the flag is false, depending on the target platform, this can lead to
the pipeline being flushed, and the cache being dirtied by code that is
ultimately not required to even be run.

%dirty flag used to build work list

This code, on most platforms, will run faster, even though it's apparently
doing more work. The difference is that this code runs through all the
necessary elements only loading the data necessary to perform the check to see
if work is required to be done. Once the work list is finished, then a definite
call to do work can be made on each item in the list. In the worst case, all
the items in the container require work, and the work list is full, and
depending on your target, this might be slower than the original iteration, but
if there will be a number of dirty items at which the two pass process is
faster, and only actually counting and profiling will find that value out. You
cannot rely on a compiler to reorganise your calls this way.

Some work has been done in virtual machines to do optimisations like this,
which leads to really good performance in spite of badly laid out code, but the
performance comes at a price, namely the fact that during runtime, the virtual
machine is having to look for optimisations to make.

\section{Domain Knowledge}

Every time you find that you're looking up the answer to a question and find
that though the calculation seems complex, the answer is highly correlative to
only a small subset of the inputs, there's a chance that you're missing some
domain knowledge. Sometimes you don't even need the answer to a question as
the design prescribes an implicit value. These implicit values, when figured
into the code, can reduce a highly complex function to one that is simple, or
better, doesn't need to run as often.

The full health bar that doesn't need regenerating is a form of domain knowledge
that lead to existence based processing of the regeneration code. Code that would
otherwise have run without effect now only runs when required.

Knowing that a compute kernel only affects elements that have differing values,
such as a Gaussian blur, can affect how much time it takes to process the data.
Marking regions of a data-store as potentially able to be affected by the
kernel in a pre-parse can make all the difference, and can also help split a
task across multiple workers.

Experimentation, documentation, and analysis, are the key ingredients to any
real understanding of data transforms. Without looking at the data that an
algorithm is using, and what comes out given that data, there will never be
any hope of insight or improvement.

In one analysis, an assumed to be completely smooth accelerometer reading was
proven to include the jerk vibrations from the force propelling the sensor.
This caused noisy data which increased the difficulty to complete the intended
analysis. Taking this base noise into account allowed for a much simpler and
higher accuracy algorithm.

When trying to find out what was causing a DVD read to run slow, an analysis
was required to find out where the head was while it was reading. In the data,
we found that the game was reading data from where we expected, but also from a
configuration file that was meant to be read in debug builds. The file wasn't
really being read, but it's date-stamp was, which meant that the read head
wasn't where it was meant to be while streaming, once a second. Without
checking the data, there wouldn't have been a very good chance to find the bug.

Sometimes domain knowledge comes purely from the design of the game. Knowing
that you can eject the data for an area of the game as a section is one way
only can be a simple memory saver. Knowing that building cannot return to their
fixed state after being destroyed is important too. For most older games that
did allow destructible environments, the knowledge that the environment didn't
need to be reversible was used to reduce the memory footprint of the level
data.  Most games would require reloading the level before being able to
restart as once the structures were destroyed, especially with procedural
destruction, there was no information on how the building started out. This
lack of information was a space saver, and could have been seen as a form of
domain knowledge.

Other key pieces of data can come from without. Analytics, both from people
playing the game and bots, can provide much needed data. Making a bot for a
single-player game can be very rewarding as it allows you to run tests for
things that would otherwise be too simple and thus too boring for even the most
dedicated testers. For example, running a frame-rate analysis for a game can be
highly tedious, but can provide a massive amount of beneficial data to both the
art team, and the engine team, as they can look at specific hotspot areas in
both the art assets, and the engine technology. Knowledge of what really
happens in the game in this case leads to optimising only the parts of the game
engine that are really required to produce the game. Without a real analysis of
the performance of the game given the real data, the real game code and a real
run through, any optimisations have to be speculative at best. Even if you can
prove that an area of the engine code is running really slowly, there's no
point in optimising it if it's only used in sections of the game where the
engine has plenty of spare resources and doesn't need to run any faster.
Another reason why running these tests via a bot is important is that as with
all optimisations and refactorings, it's important that once you have made your
changes you must be able to see the outcome of those changes, and must be able
to prove that you haven't broken anything else while changing the code, and
that you actually improved things.

Sometimes it can be hard to run the same test multiple times. Sometimes your
test isn't entirely deterministic. When that is the case you need to gather
your data more carefully and be aware of the statistical significance of any
data you have. Learn how to determine what constitutes significant values and
can be confirmed as improvement over the noise inherent in the data you can
collect.

\section{Facts from counting}

Proofs sometimes come from trying to find out the patterns behind data. Find a
pattern in your results from a function and you might find a much simpler way
to calculate your values.


%morten numbers

%graycodes

%space filling curves

%overlapping bounds
ryg

%compression techniques
simple9, run length encoding, dictionary

%runtime choices
array or map, or spatial partitioning?


