\chapter{Finite State Machines}

Finite state machines are a solid solution to many game logic problems. They
are the building block of most medium level AI code, and provide testable
frameworks for user interfaces. Notoriously convoluted finite state machines
rule the world of memory card operations and network lobbies, while simple ones
hold domain over point and click adventure puzzle logic. Finite state machines
are highly data driven in that they react to an alphabet of signals, and change
state based on both the signal received, and their current state.

\section{Tables as States}

If we wish to implement finite state machines without objects to contain them,
and without state variables to instruct their flow, we can call on the runtime
dynamic polymorphism inherent in the data oriented approach to provide
these characteristics based only on the existence of rows in tables. We do not
need an instruction cache trashing state variable that would divert the flow
according to internal state. We don't need an object to contain any state
dependent data for analysis of the alphabet.

At the fundamental level, a finite state machine requires a reaction to an
alphabet. If we replace the alphabet handling code with condition tables, then
we can produce output tables of transitions to commit. If each state is
represented by a condition table, and any entity in that state represented by a
row in a state table, then we can run each state in turn, collecting any
transitions in buffers, then committing these changes after everything has
finished a single update step.

Finite state machines can be difficult to debug due to their data-driven
nature, so being able to move to a easier to debug framework should reduce
development time. In my experience, being able to log all the transitions over
time has reduced some AI problems down to merely looking through some logs
before fixing an errant condition based only on an unexpected transion logged
with its cause data.

Keeping the state as a table entry can also let the FSM do some more advanced
work, such as managing level of detail or culling. If your renderables have a
state of potentially visible, then you can run the culling checks on just these
entities rather than those not even up for consideration this frame. Using
count lodding with FSMs allows for game flow logic as allowing the triggering
of a game state change to emit the state's linked entities could provide a good
point for debugging any problems. Also, using condition tables makes it very
easy to change the reactions and add new alphabet without having to change a
lot of code.

\section{Implementing Transitions}

Finite state machines normally have an input stream that modifies the internal
state as fast as new signals arrive. This can lead to very fast state
switching, but to maintain the parallelism, transform oriented finite state
machines only run one update based on all signals available at the time of the
tick. There is no ambiguity however, as if there are more than one conditional
responses matching the input state, then it is fine to go to more than one
state at once as a result.

Consider the finite state machine in a point and click adventure, where the
character has to find three objects. In most games, the logic would be defined
as a state waiting for all objects to be found. In a condition table finite
state machine, there could be three different tables representing each of the
objects, and while any table is populated, the game will not progress.

In a transform oriented finite state machine, the transitions are generated by
the condition tables transforming the input signals into table insertions and
deletions. Whenever a condition is matched, it emits an entity ID and a bool
into a transition table (or a null if you intend to reduce the table in a
separate process) ready for processing in the commit stage. This means that a
finite state machine can react to multiple signals in a deterministic way
because the state will not change before it has finished processing all the
possible condition matches. There is no inherent temporal coupling in this
design. Traditional finite state machines don't allow for multiple reactions at
once as they transition from one state to another, naturally reacting on the
order of signals, which is perfect for a finite state machine built for a lexer
or parser, but possibly not for a generic gameplay state machine.

\section{Condition tables as Triggers}

Sometimes a finite state machine transition has side effects. Some finite state
machines allow you to add callbacks on transitions, so you can attach onEnter,
onExit, and onTransition effects. Because the transform oriented approach has a
natural rhythm of state transforms to transition request transforms to new
state, it's simple to hook into any state transition request table and add a
little processing before the state table row modifications are committed.

You can use hooks for logging, telemetry, or game logic that is watching
certain states.  If you have a finite state machine for mapping input to player
movement, it's important to have it react to a player state that adjusts the
control method. For example, if the player has different controls when under
water, the onEnter of the inWater state table could change the player input
mapping to allow for floating up or sinking down. It would also be a good idea
to attach any oxygen-level gauge hooks here to. If there is meant to be a
renderable for the oxygen-level, it should be created by the player entering
the inWater state. It should also be hooked into the inWater onExit transition
table, as it will probably want to reset or begin regenerating at that point.

One of the greatest benefits of being able to hook into transitions, is being
able to keep track of changes in game state for syncing. If you are trying to
keep a network game sycnronised, knowing exactly what happened and why can be
the difference between a 5 minute and 5 day debugging session.

\section{Conditions as Events}

The entries in a condition table represent the events that are interesting, the
ones that affect the flow of the game and the way in which the data is
transformed. This means that the condition tables drive the flow, but they
themselves are tables that can be manipulated, so they can self modify, which
means we can use this technique to provide high perfomance data-driven, yet
data-oriented development practices.

Condition tables are tables that can have rows inserted or deleted just like
any other table. If the rows are added, modified, or removed based on the same
rules as everything else, then you have a data-driven general purpose
development model. One that doesn't even sacrifice performance for flexibility.


