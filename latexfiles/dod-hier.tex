\chapter[Hierarchical Level of Detail]{Hierarchical Level-of-Detail 
and Implicit-state}\label{chap:hier}

Consoles and graphics cards are not generally bottlenecked at the polygon
rendering stage in the pipeline. Usually they are fill rate bound if there are
large polygons on screen, or there is a lot of alpha blending, and for the most
part, graphics chips spend a lot of their time reading textures. Because of
this, the old way of doing level of detail with multiple meshes with decreasing
numbers of polygons is never going to be as good as a technique that takes into
account the actual data required of the level of detail used in each
renderable.  Hierarchical level of detail fixes the problem of high primitive
count and mistargetted art optimisations by grouping and merging many low level
of detail meshes into one low level of detail mesh, thus reducing the time
spent in the setup of render calls, and enforcing a better perspective on the
artist producing the lower resolution asset. In a typical very large scale
environment, a hierarchical level of detail implementation can reduce the
workload on a game engine by an order of magnitude as the number of entities in
the scene considered for rendering drops significantly. Even though the number
of polygons rendered might be exactly the same, or maybe even more, the fact
that the engine usually only has to handle a static number of entities at once
increases stability and allows for more accurately targeted optimisations of
both art and code.

\section[Existence]{Existence from Null to Infinity}

Going back to our entities being implicit on their attributes, we can utilise
the theory of hierarchical level of detail to offer up some optimisations for
our code. If we have high level low fidelity coarse grain game logic for distant
or currently unimportant game entities, and only drill down to highly tuned
code when an entity becomes more apparent to the player, then we can save a lot
of cycles on things the player is not interested in, or possibly even able to
see.

Consider a game where you are defending a base from incoming attacks. The
attackers come in squadons of ships, you can see them all coming at once, over
a thousand ships in all and up to twenty at once in each squadron. You have to
shoot them down, or be inundated with gunfire and bombs, taking out both you
and the base you are defending.

Running full AI, with swarming for motion and avoidance for your slower moving
weapons might be too much if it was run on all thousand ships every tick, but
you don't need to. The basic assumption made by most AI programmers is that
unless they are within attacking range, then they don't need to be running AI.
This is true, and does offer an immediate speed up compared to the naive
approach. Hierarchical LOD provides another way to think about this, through
changing the number of entities based on how they are perceived by the player.
For want of a better term, count-lodding is a name that describes what is
happening behind the scenes a little better, because sometimes there is no
hierarchy and yet there can still be a change in count between the levels of
detail.

In the count-lodding version of the base defender game, there is a few Wave
entities that project a few Squadron blips on the radar. The squadrons don't
exist as their own entities until they get close enough. Once a wave's squadron
is within range, the Wave can decrease its squadron count, and pop out a new
Squadron entity. The Squadron entity shows blips on the radar for each of its
component ships. The ships don't exist, but they are implicit in the Squadron.
The Wave continues to pop Squadrons as they come into range, and once it's
internal count has dropped to zero, it deletes itself as it now represents no
entities. As a Squadron comes into even closer range, it pops out its ships
into their own entities, and deletes itself. As the ships get closer, their
renderables are allowed to switch to higher resolution and their AI is allowed
to run at a higher intelligence setting.

When the ships are shot at, they switch to a taken damage type much like the
health system earlier. They are full health unless they take damage. If an AI
reacts to damage with fear, they may eject, adding another entity to the world.
If the wing of the plane is shot off, then that also becomes a new entity in
the world. Once a plane has crashed, it can delete its entity and replace it
with a smoking wreck entity that will be much simpler to process than an
aerodynamic simulation.

If the player can't keep the ships at bay and their numbers increase in size so
much that any normal level of detailing can't kick in, count lodding can still
help by returning ships to squadrons and flying them around the base attacking
as a group rather than as invididual ships. The level of detail heuristic can
be tuned so that the nearest and front-most squadron are always the highest
level of detail, and the ones behind the player maintain a very simplistic
representation.

This is game development smoke and mirrors as a basic game engine element. In
the past we have reduced the number of concurrent attacking AI\footnote{I
beleive this was half-life}, reduced the number of cars on screen by
staggering the lineup over the whole race track\footnote{Ridge Racer was one of
the worst for this}, and we've literally combined people together into one
person instead of having loads of people on screen at once\footnote{Populous
did this}. This kind of reduction of processing is commonplace. Now use it
everywhere appropriate, not just when a player is not looking.

\section{Mementos}

Reducing detail introduces an old problme, though. Changing level of detail in
game logic systems, AI and such, brings with it the loss of high detail
history. In this case we need a way to store what is needed to maintain a
highly cohesive player experience. If a high detail squadron in front of the
player goes out of sight and another squadron takes their place, we still want
any damage done to the first group to reappear when they come into sight again.
Imagine if you had shot out the glass on all the ships and when they came round
again, it was all back the way it was when they first arrived. A cosmetic
effect, but one that is jarring and makes it harder to suspend disbelief.

When a high detail entity drops to a lower level of detail, it should store a
memento, a small, well compressed nugget of data that contains all the
necessary information in order to rebuild the higher detail entity from the
lower detail one. When the squadron drops out of sight, it stores a memento
containing compressed information about the amount of damage, where it was
damaged, and rough positions of all the ships in the squadron. When the
squadron comes into view once more, it can read this data and generate the high
detail entities back in the state they were before. Lossy compression is fine
for most things, it doesn't matter precisely which windows, or how they were
cracked, maybe just that about $2/3$ of the windows were broken.

Another good example is in a city based free-roaming game. If AIs are allowed
to enter vehicles and get out of them, then there is a good possibility that
you can reduce processing time by removing the AIs from world when they enter a
vehicle. If they are a passenger, then they only need enough information to
rebuild them and nothing else. If they are the driver, then you might want to
create a new driver type based on some attributes of the pedestrian before
making the memento for when they exit the vehicle.

If a vehicle reaches a certain distance away from the player, then you can
delete it. To keep performance high, you can change the priorities of vehicles
that have mementos so that they try to lose sight of the player thus allowing
for earlier removal from the game. Optimisations like this are hard to
coordinate in Object-oriented systems as internal inspection of types isn't
encouraged.

\section{JIT Data Generation}

If a vehicle that has been created as part of the ambient population is
suddenly required to take on a more important role, such as the car being
involved in a fire fight. It is important to generate new entities that don't
seem overly generic or unlikely given what the player knows about the game so
far. Generating data can be thought of as providing a memento to read from just
in time. JIT mementos are faked mementos that provide a false sense of
continuity by allowing pseudo random generators or hash functions the
opportunity to replace non-zero memory usage mementos when the data is unlikely
to change, or be needed for more than a short while.

JIT mementos can also be the basis of a highly textured environment with
memento style sheets or style guides that can direct a feel bias for any
mementos generated in those virtual spaces. Imagine a city style guide that
specifies rules for occupants of cars, that businessmen might share, but are
less likely to, families have children in the back seats with mum or dad
driving, young adults tend to drive around in pairs. These style guides help
add believability to any generated data. Add in local changes such as having
types of car linked to their drivers, convertibles driven by well dressed types
or kids, low riders driven almost exclusively by ghetto residents, imports
driven by young adults. In a space game, dirty hairy pilots of cargo ships,
well turned out officers commanding yachts, rough and ready mercenaries in
everything from a single seater to a dreadnought.

JIT mementos are a good way to keep variety up, and style guides bias that so
it comes without the impression that everyone is different so everyone is the
same. When these biases are played out without being striclty adhered to, you
can build a more textured environment. If your environment is heavily populated
with a completely different people all the time, there is nothing to hold onto,
not patterns to recognise. When there are no patterns, the mind tends to see
noise, or mark it as samey. Even the most varied virtual worlds look bland when
there is too much content all in the same place.

\section{Alternative Axes}

As with all things, take away an assumption and you can find other uses for a
tool. Whenever you read about or work with a level of detail system, you will
be aware that the constraint on what level of detail is shown has always been
some distance function in space. It's now time to take that assumption, discard
it, and analyse what is really happening.

First, we find that if we take away the assumption of distance, we can infer
the conditional as some kind of linear measure. This value normally comes from
a function that takes the camera position and finds the relative distance to
the entity under consideration. What we may also realise when discarding the
distance assumption is a more fundamental understanding of that what we are
trying to do.  We are using a runtime variable to control the presentation
state of an entity. We use runtime variables to control the state of many parts
of our game already, but in this case, there is a passive presentation response
to the variable, or axis being monitored. The presentation is usually some
graphical, or logical level of detail, but it could be something as important
to the entity as its own existence.

How long until a player forgets about something that might otherwise be
important? This information can help reduce memory usage as much as distance.
If you have ever played {\em Grand Theft Auto IV}, you might have noticed that
the cars can dissappear just by not looking at them. As you turn around a few
times you might notice that the cars seem to be different each time you face
their way.  This is a stunning use of temporal level of detail. Cars that have
been bumped into or driven and parked by the player remain where they were,
because, in essence, the player put them there. Because the player has
interacted with them, they are likely to remember that they are there.
However, ambient vehicles, whether they are police cruisers, or civilian
vehicles, are less important and don't normally get to keep any special status
so can vanish when the player looks away.

In adition to time-since-seen, some elements may base their level of detail on
how far a player has progressed in the game, or how many of something a player
has, or how many times they have done it. For example, a typical bartering
animation might be cut shorter and shorter as the game uses the axis of {\em
how many recent barters} to draw back the length of any non-interactive
sections that could be caused by the event. This can be done simply, and the
player will be thankful.  It may even be possible to allow for multi-item
transactions after a certain number of transactions have happened. In effect,
you could set up gameplay elements, reactions to situations, triggers for
tutorials or extensions to gameplay options all through these abstracted level
of detail style axes.

This way of manipulating the present state of the game is safer from transition
errors. Errors that happen because going from one state to another may have set
something to true when transitioning one direction, but not back to false when
transitioning the other way. You can think of the states as being implicit on
the axis, not explicit, calculated purely as a triggered event that manipulates
state.

An example of where transition errors occur is in menu systems where though all
transitions should be reversible, sometimes you may find that going down two
levels of menu, but back only one level, takes you back to where you started.
For example, entering the options menu, then entering an adjust volume slider,
but backing out of the slider might take you out of the options menu all
together. These bugs are common in UI code as there are large numbers of
different layers of interaction. Player input is often captured in obscure ways
compared to gameplay input response. A common problem with menus is one of
ownership of the input for a particular frame. For example, if a player hits
both the forward and backward button at the same time, a state machine UI might
choose to enter whichever transition response comes first. Another might manage
to accept the forward event, only to have the next menu accept the back event,
but worst of all might be the unlikely but seen in the wild, menu transitioning
to two different menus at the same time. Sometimes the menu may transition due
to external forces, and if there is player input captured in a different thread
of execution, the game state can become disjoint and unresponsive. Consider a
network game's lobby, where if everyone is ready to play, but the host of the
game disconnects while you are entering into the options screen prior to game
launch, in a traditional state machine like approach to menus, where should the
player return to once they exit the options screen? The lobby would normally
have dropped you back to a server search screen, but in this case, the lobby
has gone away to be replaced with nothing. This is where having simple axes
instead of state machines can prove to be simpler to the point of being less
buggy and more responsive.


