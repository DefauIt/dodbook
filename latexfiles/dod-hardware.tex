\chapter[Hardware]{Looking at hardware}

The first thing a good software engineer does when starting work on a new
platform is read the contents listings in all the hardware manuals. The second
thing is usually try to get hello world up and running. It's uncommon for a
games development software engineer to decide it's a good idea to read all the
documentation available. When they do, they will be reading them literally, and
still probably not getting all the necessary information. When it comes to
understanding hardware, there is the theoretical restrictions implied by the
comments and data sheets in the manuals, but there is also the practical
restrictions that can only be found through working with the hardware at an
intimate level.

As most of the contemporary hardware is now API driven, with hardware manuals
only being presented to the engineers responsible for graphics, audio, and
media subsystems, it's tempting to start programming on a new piece of hardware
without thinking about the hardware at all. Most programmers working on big
games in big studios don't really know what's going on at the lower levels of
their game engines they're working on, and to some extent that's probably good
as it frees them up to write more gameplay code, but there comes a point in
every developer's life when they have to bite the bullet and find out why their
code is slow. Some day, you're going to be five weeks from ship and need to
claw back five frames a second on one level of the game that has been optimised
in every other area other than yours. When that day comes, you'd better know
why your code is slow, and to do that, you have to know what the hardware is
doing when it's executing your code.

Some of the issues surrounding code performance are relevant to all hardware
configurations. Some are only pertinent to configurations that have caches, or
do write combining, or have branch prediction, but some hardware configurations
have very special restrictions that can cause odd, but simple to fix
performance glitches caused by decisions made during the chip's design process.
These glitches are the gotchas of the hardware, and as such, need to be learnt
in order to be avoided.

When it comes to the overall design of console CPUs, The XBox360 and PS3 are
RISC based, low memory speed, multi-core machines, and these do have a set of
considerations that remain somewhat misunderstood by mainstream game
developers. Understanding how these machines differ from the desktop x86
machines that most programmers start their development life on, can be highly
illuminating. The coming generation of consoles and other devices will change
the hardware considerations again, but understanding that you do need to
consider the hardware can sometimes only be learned by looking at historic
data.

\section{Sequential data}

When you process your data in a sequence, you have a much higher chance of
cache hit on reading in the data. Making all your calculations run from
sequential data not only helps hardware with caches for reading, but also when
using hardware that does write combining.

In theory, if you're reading one byte at a time to do something, then you can
almost guarantee that the next byte will already be in memory the next 127
times you look. For a list of floats, that works out as one memory load for 32
values. If you have an animation that has less than 32 keys per bone, then you
can guarantee that you will only need to load as many cache-lines as you have
bones in order to find all the key indexes into your arrays of transforms.

In practice, you will find that this only applies if the process you run
doesn't load in lots of other data to help process your stream. That's not to
say that trying to organise your data sequentially isn't important, but that
it's just as important to ensure that the data being accessed is being accessed
in patterns that allow the processors to leverage the benefits of that form.
There is no point in making data sequential if all you are going to do is use
it so slowly that the cache fills up between reads.

Sequential data is also easier to split among different processors as there is
little to no chance of cache sharing. When your data is stored sequentially,
rather than randomly, you know where in memory the data is, and so you can
dispatch tasks to work on guaranteed unshared cache-lines. When multiple CPUs
compete to write to a particular cache-line, they have to storing and loading
to keep things consistent. If the data is randomly placed, such as when you
allocate from a memory pool, or directly from the heap, you cannot be sure what
order the data is in and can't even guarantee that you're not asking two
different CPUs to work on the same cache-line of data.

The data-oriented approach, even when you don't use structs of arrays, still
maintains that sequential data is better than random allocations. Not only is
it good for the hardware, it's good for simplicity of code as it generally
promotes transforms rather than object-oriented messaging.

\section{Deep Pipes}

CPUs perform instructions in pipelines. This is true of all processors,
however, the number of stages differs wildly. For games developers, it's
important to remember that it affects all the CPUs they work on, from the
current generation of consoles such as Sony's PS3 and Microsoft's Xbox360, but
also to hand helds such as the Nintendo DS, the iPhone, and other devices.

Pipelines provide a way for CPUs to trade gains in speed for latency and branch
penalties. A non-pipelined CPU finishes every instruction before it begins the
next, however a pipelined CPU starts instructions and doesn't necessarily
finish them until many cycles later. If you imagine a CPU as a factory, the
idea is the equivalent of the production line, where each worker has one job,
rather than each worker seeing and working on a product from start to finish. A
CPU is better able to process more data faster this way because by increasing
the latency, in well thought out programs, you only add a few cycles to any
processing during prologue or epilogue. During the transform, latency can be
mitigated by doing more work on non-related data while waiting for any
dependencies. Because the CPUs have to do a lot less per cycle, the cycles take
less time, which is what allows CPUs to get faster. What's happening is that it
still takes just as long for a CPU to do an operation as it always has (give or
take), but because the operation is split up into a lot of smaller stages, it
is possible to do a lot more operations per second as all of the separate
stages can operate in parallel, and any efficient code concentrates on doing
this after all other optimisations have been made.

When pipelining, the CPU consists of a number of stages, firstly the fetch and
decode stages, which in some hardware are the same stage, then an execute stage
which does the actual calculation. This stage can take multiple cycles, but as
long as the CPU has all the cycles covered by stages, it won't affect
throughput. The CPU then finally stores the result in the last stage, dropping
the value back into the output register or memory location.

With instructions having many stages, it can take many cycles for them to
complete, but because only one part of the instruction is in use at each stage,
a new instruction can be loaded as soon as the first instruction has got to the
second stage. This pipe-lining allows us issue many more instructions than we
could otherwise, even though they might have high latency in themselves, and
saves on transistor count as more transistors are being used at any one time.
There is less waste. In the beginning, the main reason for this was that the
circuits would take a certain amount of time to stabilise. Logic gates, in
practice, don't immediately switch from one logic state to another. If you add
in noise, resonance, and manufacturing error, you can begin to see that CPUs
would have to wait quite a while between cycles, massively reducing the CPU
frequency. This is why FPGAs cannot easily run at GHz speeds, they are arrays
of flexible gate systems, which means that they suffer the most from stability
problems, but amplified by the gates being located a long way from each other,
in the sense that they are not right up next to each other like logic circuits
are inside an inflexible ASIC like a production CPU.

Pipelines require that the instructions are ready. This can be problematic if
the data the instruction is waiting on is not ready, or if the instruction is
not loaded. If there is anything stopping the instruction from being issued it
can cause a stall, or in the case of branching causing the instructions to be
run, then trashed as the pipe-line is flushed ready to begin processing the
correct branch. If the instruction pointer is determined by some data value, it
can mean a long wait while the next instruction is loaded from main memory. If
the next instruction is based on a conditional branch, then the branch has to
wait on the condition, thus causing a number of instructions to begin
processing when the branch could invalidate all the work done so far. As well
as instructions needing to be nearby, the registers must already be populated,
otherwise something will have to wait.

\section[Microcode]{Microcode: virtually function calls.}

To get around the limitations implicit in trying to increase throughput, some
instructions on the RISC chips aren't really there. Instead, these virtual
instructions are like function calls, calls to macros that run a sequence of
instructions. These instructions are said to be micro-coded, and in order to
run, they often need to commandeer the CPU for their entire duration to
maintain atomicity. Some functions are micro-coded due to their infrequent use
or relative cost to implement as an intrinsic instruction, some because of the
spec, and some because they don't fit well with the pipe-lined model of
execution. In all of these cases, a micro-coded instruction causes a gap,
called a bubble, in the pipeline, and that's wasted execution time.  In almost
all cases, these microcoded instructions can be avoided, sometimes by changing
command line parameters (ref Cell Performance), sometimes by adjusting how you
solve a problem (ref 1<<n hack), and sometimes by changing the problem
completely. (sqrt considered harmful)

\section{Single Instruction Multiple Data}

There are no current generation consoles that don't have SIMD of some sort. All
hardware now has some kind of vector unit, and to some extent, as long as you
work within your boundaries, even hardware that doesn't have SIMD instructions,
such as embedded micro controllers, can operate on multiple data.  The idea
behind SIMD is simple: issue one command, and manipulate multiple pieces of
data in the same way at the same time. The most commonly referenced
implementation of this is the vector units inherent in all current generation
hardware. The AlitVec instructions on PPC and the SPU instruction set contain
many instructions that operate on multiple pieces of data at the same time,
sometimes doing asymetric operations such as rotating, splatting, or
reconfiguring the vectors. On older machines or simple machines, the explicit
instructions may not exist, but in the world of bitwise logic, we've always had
some SIMD instructions hanging around as all the bitwise ops run over multiple
elements in a bit field of whatever native word length. Consider some of the
winners of the quickest bit counting routines. My favourite is the purely SIMD
style bit counter given here:

\begin{lstlisting}
uint32_t CountBits( uint32_t in ) {
	v = v - ((v >> 1) & 0x55555555);                    // reuse input as temporary
	v = (v & 0x33333333) + ((v >> 2) & 0x33333333);     // temp
	c = ((v + (v >> 4) & 0xF0F0F0F) * 0x1010101) >> 24; // count
	return c;
}
\end{lstlisting}

So, look to your types, and see if you can add a bit of SIMD to your
development without even breaking out the vector instrinsics.

\section{Predictable instructions}

The biggest crime to commit in a deeply pipelined core is to tell it to do
loads of instructions, then once it's almost done, change your mind and start
on something completely different. This heinous crime is all too common, with
control flow instructions doing just that when they're hard to predict, or
impossible to predict in the case of entirely random data, or where the data
pattern is known, but the architecture doesn't support branch predictions.

