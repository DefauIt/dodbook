\chapter{Existence Based Processing}

If you saw that there weren't any apples in stock, would you still haggle over
their price?

\section{Why use an if}\label{sec:exist-whyif}

When studying software engineering you may find reference to cyclomatic
complexity or conditional complexity. This is a complexity metric providing a
numeric representation of the complexity of code, and is used in analysing
large scale software projects. Cyclomatic complexity concerns itself only with
flow control. The formula, summarised for our purposes, is one (1) plus the
number of conditionals present in the system being analysed. That means for any
system it starts at one, and for each if, while, for, and do-while, we add one.
We also add one per path in a switch statement excluding the default case if
present (this is because whether or not the default case is included, it is
part of the possible routes through the code, so counts like the else case of
an if, as in, it isn't counted, only the met conditions are counted.) under the
hood, if we consider how a virtual call works, that is, a lookup in a function
pointer table followed by a branch into the class member function, we can see
that a virtual call is effectively just as complex as a switch statement.
Counting the flow control statements is more difficult in a virtual call
because to know the complexity value, you have to know the number of possible
classes member functions that can fulfil that request. In the case of a virtual
call, you have to count the number of overrides to a base virtual call. If the
base is pure-virtual, then you may subtract one from the complexity. However,
if you don't have access to all the code that is running, which can be possible
in the case of dynamically loaded libraries, then the number of different code
paths that the process can take increases by an unknown amount. This hidden or
obscured complexity is necessary to allow third party libraries to interface
with the core process, but requires a level of trust that implies that no
single part of the process is ever going to be thoroughly tested.

Analysing the complexity of a system helps us understand how difficult it is to
test, and in turn, how hard it is to debug. Sometimes, the difficulty we have
in debugging comes from not fully observing all the flow control points, at
which point the program will have entered into a state we had not expected or
prepared for. With virtual calls, the likelihood of that happening can
dramatically increase as we can not be sure that we know all the different ways
the code can branch until we either litter the code with logging, or step
through in a debugger to see where it goes at run-time.

Returning to complexity in general, we must submit that we use flow control to
change what is executed in our programs. In most cases these flow controls are
put in for one of two reasons: design, and implementation necessity. The first
is when we need to implement the design, a gameplay feature that has to only
happen when some conditions are met, such as jumping when the jump button is
pressed, or autosaving at a save checkpoint when the savedata is dirty. The
second form of flow control is structural, or defensive programming techniques
where the function operating on the data isn't sure that the data exists, or is
making sure bounds are observed.

In real world cases, the most common use of an explicit flow control statement
is in defensive programming. Most of our flow control statements are just to
stop crashes, to check bounds, pointers being null, or any other exceptional
cases that would bring the program to a halt. The second most common is loop
control, though these are numerous, most CPUs have hardware optimisations for
these, and most compilers do a very good job of removing condition checks that
aren't necessary. The third most common flow control comes from polymorphic
calls, which can be helpful in implementing some of the gameplay logic, but
mostly are there to entertain the do-more-with-less-code development model
partially enforced in the object-oriented approach to writing games. Actual
visible game design originating flow control comes a distant fourth place in
these causes of branching, leading to an under-appreciation of the effect each
conditional has on the performance of the software. That is, when the rest of
your code-base is slow, it's hard to validate writing fast code for any one
task.

If we try to keep our working set of data as a collections of arrays, we can
guarantee that all our data is not null. That one step alone will eliminate
most of our flow control statements. The third most common set of flow control
statements were the inherent flow control in a virtual call, they are covered
later in section \ref{sec:exist-poly}, but simply put, if you don't have an
explicit type, you don't need to switch on it, so those flow control statements
go away as well. Finally, we get to gameplay logic, which there is no simple
way to eradicate. We can get most of the way there with condition tables which
will be covered in chapter \ref{chap:condition}, but for now we will assume
they are allowed to stay.

Reducing the amount of conditions, and thus reducing the cyclomatic complexity
on such a scale is an amazing benefit that cannot be overlooked, but it is one
that comes with a cost. The reason we are able to get rid of the check for null
is that we now have our data in a format that doesn't allow for null. This
inflexibility will prove to be a benefit, but it requires a new way of
processing our entities. Where we once had rooms, and we looked in the rooms to
find out if there were any doors on the walls we bumped into (in order to
either pass through, or instead do a collision response,) we now look in the
table of doors to see if there are any that match our roomid. This reversal of
ownership can be a massive benefit in debugging, but sometimes can appear
backwards when all you want to do is find out what doors you can use to get out
of a room.

If you've ever worked with shopping lists, or todo lists, you'll know how much
more efficient you are when you have a definite list of things to get or get
done. It's very easy to make a list, and easy to add to it too. If you're going
shopping, it's very hard to think what might be missing from your house in
order to get what you need. If you're the type that tries to plan meals, then a
list is nigh on essential as you figure out ingredients and then tally up the
number of tins of tomatoes, or how many different meats or vegetables you need
to last through all the meals you have planned. If you have a todo list and a
calendar, you know who is coming and what needs to be done to prepare for them,
how many extra mouths need feeding, how many extra bottles of beer to buy, or
how much washing you need to do to make enough beds for the visitors. Todo
lists are great because you can set an end goal and then add in sub tasks that
make a large and long distant goal seem more doable, and also provide a little
urgency that is usually missing when the deadline is so far away.

When your program is running, if you don't give it lists to work with, and
instead let it do whatever comes up next, it will be inefficient, slow, and
probably have irregular frame timings. Inefficiency comes from not knowing what
kind of processing is coming up next. In the case of large arrays of pointers
to heterogeneous classes all being called with an \texttt{update()} function,
you can get very high counts of cache misses both in data and instruction
cache. If the code tries to do a lot with this pointer, which it usually does
because one of the major beliefs of object-oriented programmers is that virtual
calls are only for higher level operations, not small, low level ones, then you
can virtually guarantee that not only will the instruction and data cache be
thrashed by each call, but most branch predictor slots could be too dirty to
offer any benefit when the next \texttt{update()} runs. Assuming that virtual
calls don't add up because they are called on high level code is fine until
they become the go to programming style and you stop thinking about how they
affect your application when there are millions of virtual calls per second.
All those inefficient calls are going to add up somewhere, but luckily for the
object oriented zealot, they never appear on any profiles. They always appear
somewhere in the code that is being called.  Slowness also comes from not being
able to see how much work needs to be done, and therefore not being able to
scale the work to fit what is possible in the time-frame given. Without a todo
list, and an ability to estimate the amount of time that each task will take,
it is impossible to decide the best course of action to take in order to reduce
overhead while maintaining feedback to the user. Irregular frame timings also
can be blamed on not being able to act on distant goals ahead of time. If you
know you have to load an area because a player has ventured into a position
where it is possible they will soon be entering an unloaded area, the streaming
system can be told to drag in any data necessary. In most games this happens
with explicit triggers, but there is no such system for many other game
elements. It's unheard of for an AI to pathfind to some goal because there
might soon be a need to head that way. It's not commonplace to find a physics
system doing look ahead to see if a collision has happened in the future in
order to start doing a more complex breakup simulation. But, if you let your
game generate todo lists, shopping lists, distant goals, and allow for
preventative measures by forward thinking, then you can simplify your task as a
coder into prioritising goals and effects, or writing code that generates
priorities at runtime.

In addition to the obvious lists, metrics on your game are highly important. If
you find that you've been optimising your game for a silky smooth frame rate
and you think you have a really steady 30fps or 60fps, and yet your customers
and testers keep coming back with comments about nasty frame spikes and
dropout, then you're not profiling the right thing. Sometimes you have to
profile a game while it is being played. Get yourself a profiler that runs all
the time, and can report the state of the game when the frame time goes over
budget. Sometimes you need the data from a number of frames around when it
happened to really figure out what is going on, but in all cases, unless you're
letting real testers run your profiler, you're never going to get real world
profiling data.

Existence-based-processing is when you process every element in a homogeneous
set of data. You run the same instructions for every element in that set.
There is no definite requirement for the output in this specification, however,
usually it is one of three types of operation: filter, mutation, or emission.
A mutation is a one to one manipulation of the data, it takes incoming data
and some constants that are setup before the transform, and produces one
element for each input element. A filter takes incoming data, again with some
constants set up before the transform, and produces one element or zero
elements for each input element. An emission is a manipulation on the incoming
data that can produce multiple output elements. Just like the other two
transforms, an emission can use constants, but there is no guaranteed size of
the output table; it can produce anywhere between zero and infinity elements.

Every CPU can efficiently handle running processing kernels over homogeneous
sets of data, that is, doing the same operation over and over again over
contiguous data. When there is no global state, no accumulator, it is proven to
be parallelisable. Examples can be given from existing technologies such as
mapreduce and opencl as to how to go about building real work applications
within these restrictions. Stateless transforms also commit no crimes that
prevent them from being used within distributed processing technologies. Erlang
relies on these as language features to enable not just thread safe processing,
not just inter process safe processing, but distributed computing safe
processing. Stateless transforms of stateful data is highly robust, and deeply
parallelisable.

Within the processing of each element, that is for each datum operated on by
the transform kernel, it is fair to use control flow. Almost all compilers
should be able to reduce simple local value branch instructions into a
platform's preferred branchless representation. When considering branches
inside transforms, it's best to compare to existing implementations of stream
processing such as graphics card shaders or OpenCL kernels.

In predication, flow control statements are not nearly ignored, they are used
instead as an indicator of how to merge two results. When the flow control is
not based on a constant, a predicated if will generate code that will run both
sides of the branch at the same time and discard one result based on the value
of the condition. It manages this by selecting one result based on the
condition, in some CPUs there is an fsel intrinsic, other CPUs may have a cmov
instruction, but all CPUs can use masking to effect this trick.

There are other solutions in the form of SIMD and MIMD. SIMD or
single-instruction-multiple-data allows the parallel processing of data when
the instructions are the same. If there are any conditionals, then as long as
the result of the condition is the same across the data, the function returns
quickly. If the condition result is different across the different data, the
function stalls for one branch remembering which of the data had chosen which
path, completes for one side, then goes back and continues for the other side.
This costs time, but is not as expensive as predication as it does not always
run both branches.  In other systems, the condition can lead to new micro jobs
handed off to different cores, converging back to the main thread only once all
branches have completed. 

In MIMD, that is multiple instruction, multiple data, every piece of data can
be operated on by a different set of instructions. Each piece of data can take
a different path. This is the simplest to code for because it's how most
parallel programming is currently done. We add a thread and process some more
data with a separate thread of execution. MIMD includes multi-core general
purpose CPUs. MIMD often allows shared memory access and all the
synchronisation issues that come with it. MIMD is by far the easiest to get up
and running, but it is also the most prone to the kind of rare fatal error that
costs a lot of time to debug.

\section{Don't use booleans}\label{sec:exist-bool}

When you study compression technology, one of the most important aspects you
have to understand is the different between data and information. There are
many ways to store information in systems, from literal strings that can be
parsed to declare that something exists, right down to something simple like a
single bit flag to show that a thing might have an attributes. Examples include
the text that declares the existence of a local variable in a scripting
language, or the bit field that contains all the different collision types a
physics mesh will respond to. Sometimes we can store even less information than
a bit by using advanced algorithms such as arithmetic encoding, or by utilising
domain knowledge. Domain knowledge normalisation applies in most games
development, but it is very infrequently applied. As information is encoded in
data, and the amount of information encoded can be amplified by domain
knowledge, it's important that we begin to see the advice offered by
compression techniques is that: all we are really encoding is probabilities.

If we take an example, a game where the entities have health, regenerate after
a while of not taking damage, can die, can shoot each other, then lets see what
domain knowledge can do to reduce processing.

We assume this domain knowledge: if you have full health, then you don't need
to regenerate.  Once you have been shot, it takes some time until you begin
regenerating.  Once you are dead, you cannot regenerate.  Once you are dead you
have zero health.

If we have a table for the entity thus:

\begin{lstlisting}[caption=naive entity table]
struct entity {
	// information about the entity position
	// ... 
	// now health data in the middle of the entity
	float timeoflastdamage;
	float health;
	// ...
	// other entity information
};
list<entity> entities;
\end{lstlisting}

Then we can run an update function over the table that might look like this:

\begin{lstlisting}[caption=every entity health regen]
void updatehealth( entity *e ) {
	timetype timesincelastshot = e->timeoflastdamage - currenttime;
	bool ishurt = e->health < max_health && e->health > 0;
	bool regencanstart = timesincelastshot > time_before_regenerating;
	if( ishurt && regencanstart ) {
		e->health = min(max_health, e->health + ticktime * regenrate);
	}
}
\end{lstlisting}

Which will run for every entity in the game, every update.

We can make this better by looking at the flow control statement. The function
only needs to run if the health is less than full health, and more than zero.
The regenerate function only needs to run if it has been long enough since the
last damage dealt.

Let's change the structures:


\begin{lstlisting}[caption=existence based processing style health]
struct entity {
	// information about the entity position
	// ...
	// other entity information
};
struct entitydamage {
	float timeoflastdamage;
	float health;
}
list<entity> entities;
map<entityref,entitydamage> entitydamages;
\end{lstlisting}

We can now run the update function over the health table rather than the
entities.

\begin{lstlisting}[caption=every entity health regen]
void updatehealth() {
	foreach( eh in entitydamages ) {
		entityref entity = eh->first;
		entitydamage &ed = eh->second;
		if( ed.health < 0 ) {
			deadentities.insert( entity );
			discard(eh);
		} else {
			timetype timesincelastshot = eh->timeoflastshot - currenttime;
			bool regencanstart = timesincelastshot > time_before_regenerating;
			if( regencanstart )
				eh->health =eh->health + ticktime * regenrate;
			if( eh->health > max_health ) 
				discard(eh);
		}
	}
}
\end{lstlisting}

We only add a new entityhealth element when an entity takes damage. If an
entity takes damage when it already has an entityhealth element, then it can
update the health rather than create a new row, also updating the time damage
was last dealt. If you want to find out someone's health, then you only need to
look and see if they have an entityhealth row, or if they have a row in
deadentities table. The reason this works is that an entity has an implicit
boolean hidden in the row existing in the table. For the entityhealth table,
that implicit boolean was ishurt from the first function. For the deadentities
table, the implicit boolean of isdead, also implies a health value of 0, which
can reduce processing for many other systems. If you don't have to load a float
and check that it is less than 0, then you're saving a floating point
comparison or conversion to boolean.

Other similar cases include weapon reloading, oxygen levels when swimming,
anything that has a value that runs out, has a maximum, or has a minimum. Even
things like driving speeds of cars. If they are traffic, then they will spend
most of their time driving at {\em traffic speed} not some speed that they need
to calculate. If you have a group of people all heading in the same direction,
then someone joining the group can be {\em intercepting} until they manage to,
at which point they can give up their self and become controlled by the group.

As an aside, we use a map and a list here for simplicity of reading, and though
they are not the most optimal containers in that they are trees, or pointers to
elements, rather than a nice contiguous array of some sort, they're not going
to adversely affect the profile as much as the problem being overcome here,
namely the mixing of hot\footnote{hot data is the member or members of a class
that are accessed with high frequency, and there can even be multiple per
class.} data into the middle of the entity class.

Another example is with AI. If you have all your entities maintain a team
index, then you have to check each entity before reacting to them. If you want
to avoid all the entities from team x, and head towards the closest member of
team y, then if each team is in a different table you can just operate on all
the entities in those tables. If you want to produce an avoidance vector from
team x, you create a mapreduce function that maps team x members to an
avoidance vector, then reduce by summing.

\begin{lstlisting}[caption=map and reduce]
vector mapavoidance( entity *e ) {
	vector difference = e->pos - currentpos;
	float oneoversqr = 1.0f / difference.getsquaredlength();
	return difference * oneoversqr;
}
vector reduceavoidance( const vector l, const vector r ) {
	return l + r;
}
pair<float,entity*> mapnearest( entity *e ) {
	vector difference = e->pos - currentpos;
	float squaredistance = difference.getsquaredlength();
	return pair<float,entity*>( squaredistance, e );
}
pair<float,entity*> reducenearest( pair<float,entity*> l, pair<float,entity*> r ) {
	if( l.left < r.left ) return l; return r;
}
\end{lstlisting}

The implicit boolean in these tables is that they are worth avoiding, and that
they are worth aiming towards. If nothing maps, then the seeding value to the reduce is returned.

Another use is in state management. If an AI hears gunfire, then they can add a
row to a table for when they last heard gunfire, and that can be used to
determine whether they are in a heightened state of awareness. If an AI has
been involved in a transaction with the player, it is important that they
remember what has happened as long as the player is likely to remember it. If
the player has just sold an AI their +5 longsword, it's very important that the
shopkeeper AI still have it in stock if the player just pops out of the shop
for a moment. Some games don't even keep inventory between transactions, and
that can become a sore point if they accidentally sell something they need and
then save their progress.

The general concept of tacking on data, or patching loaded data with dynamic
additional attributes, has been around for quite a while. Save games often
encode the state of a dynamic world as a delta from the base state, and one of
the first major uses was in fully dynamic environments, where a world is
loaded, but can be destroyed or altered afterwards. Some world generators took
a procedural landscape and allowed their content creators to add patches of
extra information, villages, forts, outposts, or even break out landscaping
tools to drastically adjust the generated data. Taking that patching and
applying it to in-game runtime information adds a level of control not normally
available without using intrusive techniques.

\section{Don't use enums}\label{sec:exist-enum}

Enumerations are used to define sets of states. We could have had a state
variable for the regenerating entity, one that had infullhealth, ishurt, isdead
as its three states. We could have had a team index variable for the avoidance
entity enumerating all the available teams. Instead we used tables to provide
all the information we needed. Any enum can be emulated with a variety of
tables. All you need is one table per enumerable value. Setting the enumeration
is an insert into a table.

When using tables to replace enums, some things become more difficult: finding
out the value of an enum in an entity is difficult as it requires checking all
the tables that represent that state for the entity. However, this is mostly
disallowed and unnecessary, as the main reason for getting the value is either
to do an operation based on the state, or to find out if an entity is in the
right state to be considered for an operation.

If the enum is a state or type enum previously handled by a switch or virtual
call, then we don't need to look up the value, instead we change the way we
think about the problem. The solution is to run transforms taking the content
of each of the switch cases or virtual methods as the operation to apply to the
appropriate table, the table corresponding to the original enumeration value.

If the enum is instead used to determine whether or not an entity can be
operated upon, then you can operate on all the entities in the appropriate
table. Transforming the whole table or the join of that table with some other
condition. If you're thinking about the case where you have an entity as the
result of a query and need to know if it is in a certain state before deciding
commit some change, consider that the table you need access to could have been
one of the initial mappings, meaning that no entity would be found that didn't
match the enum replacing table in the way you needed it to.

\section{Prelude to polymorphism}\label{sec:exist-prelpoly}

Let's consider now how we implement polymorphism. We know we don't have to use
a virtual table pointer; we could use an enum as a type variable. That
variable, the member of the structure that defines at runtime what that
structure should be capable of and how it is meant to react. That variable
could be used to direct the choice of functions called when methods are called
on the object.

When your type is defined by a member type variable, it's usual to implement
virtual functions as switches based on that type, or as an array of functions.
If we want to allow for runtime loaded libraries, then we would need a system
to update which functions are called. The humble switch is unable to
accommodate this, but the array of functions could be modified at runtime.

We have a solution, but it's not elegant, or efficient. The data is still in
charge of the instructions, and we suffer the same instruction cache hit
whenever a virtual function is unexpected. However, when we don't really use
enums, but instead tables that represent each possible value of an enum, it is
still possible to keep compatible with dynamic library loading the same as with
pointer based polymorphism, but we also gain the efficiency of a data-flow
processing approach to processing heterogeneous types.

For each class, instead of a class declaration, we have a factory that produces
the correct selection of table insert calls. Instead of a polymorphic method
call, we utilise existence based processing. Our elements in a tables allow the
characteristics of the class to be implicit. Creating your classes with
factories can easily be extended by runtime loaded libraries. Registering a new
factory should be simple as long as there is a data driven factory method. The
processing tables and their \texttt{update()} functions would also be added to
the main loop.

\section{Dynamic runtime polymorphism}\label{sec:exist-poly}

If you create your classes by composition, and you allow the state to change by
inserting and removing from tables, then you also allow yourself access to
dynamic runtime polymorphism.

Polymorphism is the ability for an instance in a program to react to a common
entry point in different ways due only to the nature of the instance. In C++,
compile time polymorphism can be implemented through templates and overloading.
Runtime polymorphism is the ability for a class to provide a different
implementation for a common base operation with the class type unknown at
compile time. C++ handles this through virtual tables, calling the right
function at runtime based on the type hidden in the virtual table pointer at
the start of the memory pointed to by the this pointer. Dynamic runtime
polymorphism is when a class can react differently to a common call signature
in different ways based on both it's type and any other internal state. C++
doesn't implement this explicitly, but if a class allows the use of an internal
state variable or variables, it can provide differing reactions based on that
state as well as the core language runtime virtual table lookup. Consider the
following code:

\begin{lstlisting}[caption=simple object-oriented shape code]
class shape {
public:
	shape() {}
	virtual ~shape() {}
	virtual float getarea() const = 0;
};
class circle : public shape {
public:
	circle( float diameter ) : d(diameter ) {}
	~circle() {}
	float getarea() const { return d*d*pi/4; }
	float d;
};
class square : public shape {
public:
	square( float across ) : width( across ) {}
	~square() {}
	float getarea() const { return width*width; }
	float width;
};
void test() {
	circle circle( 2.5f );
	square square( 5.0f );
	shape *shape1 = &circle, *shape2 = &square;
	printf( "areas are %f and %f\n", shape1->getarea(), shape2->getarea() );
}
\end{lstlisting}

Allowing the objects to change shape during their lifetime requires some
compromise in C++ one way is to keep a type variable inside the class.

\begin{lstlisting}[caption=ugly internal type code]
enum shapetype { circletype, squaretype };
class mutableshape {
public:
	mutableshape( shapetype type, float argument )
		: m_type( type ), distanceacross( argument )
		{}
	~mutableshape() {}
	float getarea() const {
		switch( m_type ) {
			case circletype: return distanceacross*distanceacross*pi/4;
			case squaretype: return distanceacross*distanceacross;
		}
	}
	void setnewtype( shapetype type ) {
		m_type = type;
	}
	shapetype m_type;
	float distanceacross;
};
void testinternaltype() {
	mutableshape shape1( circletype, 5.0f );
	mutableshape shape2( circletype, 5.0f );
	shape2.setnewtype( squaretype );
	printf( "areas are %f and %f\n", shape1.getarea(), shape2.getarea() );
}
\end{lstlisting}

A better way is to have a conversion function to handle each case.

\begin{lstlisting}[caption=convert existing class to new class]
square squarethecircle( const circle &circle ) {
	return square( circle.d );
}
void testconvertintype() {
	circle circle( 5.0f );
	square square = squarethecircle( circle );
}
\end{lstlisting}

Though this works, all the pointers to the old class are now invalid. Using
handles would mitigate these worries, but add another layer of indirection in
most cases, dragging down performance even more.

If you use existence-based-processing techniques, your classes defined by the
tables they belong to, then you can switch between tables at runtime. This
allows you to change behaviour without any tricks, without any overhead of a
union to carry all the differing data around for all the states that you need.
If you compose your class from different attributes and abilities then need to
change them post creation, you can. Looking at it from a hardware point of
view, in order to implement this form of polymorphism you need a little extra
space for the reference to the entity in each of the class attributes or
abilities, but you don't need a virtual table pointer to find which function to
call. You can run through all entities of the same type increasing cache
effectiveness, even though it provides a safe way to change type at runtime.

As another potential benefit, the implicit nature of having classes defined by
the tables they belong to, there is an opportunity to register a single entity
with more than one table. This means that not only can a class be dynamically
runtime polymorphic, but it can also be multimorphic in the sense that it can
be more than one class at a time. A single entity might react in two different
ways to the same trigger call because that might be appropriate for that
current state for that class. This kind of multidimensional classing doesn't
come up much in gameplay code, but in rendering, there are usually a few
different axes of variation such as the material, what blend mode, what kind of
skinning or other vertex adjustments are going to take place on a given
instance. Maybe we don't see this flexibility in gameplay code because it's not
available through the natural tools of the language.

\section{Event handling}\label{sec:exist-event}

When you wanted to listen for events in a system in the old days, you'd attach
yourself to an interrupt. Sometimes you might get to poke at code that still
does this, but it's normally reserved for old or microcontroller scale
hardware. The idea was simple, the processor wasn't really fast enough to poll
all the possible sources of information and do something about the data, but it
was fast enough to be told about events and process the information as and when
it arrived. Event handling in games has often been like this, register yourself
as interested in an event, then get told about it when it happens. The publish
and subscribe model has been around for many years, but there's not been a
standard interface built for it as it often requires from problem domain
knowledge to implement effectively. Some systems want to be told about every
event in the system and decide for themselves, such as windows event handling.
Some systems subscribe to very particular events but want to react to them as
soon as they happen, such as handlers for the BIOS events like the keyboard
interrupt.

Using your existence in a table as the registration technique makes this
simpler than before and lets you register and de-register with great pace. You
can register an entity as being interested in events, and choose to fire off
the transform immediately, or queue up new events until the next loop round. As
the model becomes simpler and more usable, the opportunity for more common use
leads us to new ways of implementing code traditionally done via polling.

For example: unless the player character is within the distance to activate a
door, the event handler for the player's action button wont be attached to
anything door related. When the character comes within range, the character
registers into the \emph{has\_pressed\_action} event table with the
\emph{open\_door\_(X)} event result. This reduces the amount of time the
CPU wastes figuring out what thing the player was trying to activate, and also
helps provide state information such as on-screen displays saying that
\emph{pressing Green will Open the door}. It may even do this by hooking into
low level tables generated by default such as a \emph{character registers into
the has\_pressed\_action event table} event.

This coding style is somewhat reminiscent of aspect oriented programming where
it is easy to allow for cross-cutting concerns in the code. In aspect oriented
programming, the core code for any activities is kept clean, and any side
effects or vetoes of actions are handled by other concerns hooking into the
activity from outside. This keeps the core code clean at the expense of not
knowing what is really going to be called when you write a line of code. How
using registration tables differs is in where the reactions come from and how
they are determined. As we shall see in chapter \ref{chap:condition}, when you
use conditions for logical reactions, debugging can become significantly
simpler as the barriers between cause and effect normally implicit in aspect
oriented programming are significantly diminished or removed, and the hard to
adjust nature of object oriented decision making can be softened to allow your
code to become more dynamic without the normal associated cost of data driven
control flow.


