\chapter{Sorting}

For some subsystems, sorting is a highly important function. Sorting the
primitive render calls so that they render front to back for opaque objects
can have a massive impact on GPU performance, so it's worth doing. Sorting the
primitive render calls so they render back to front for alpha blended objects
is usually a necessity. Sorting sound channels by their amplitude over their
sample position is a good indicator of priority.

Whatever you need to sort for, make sure you need to sort first, as usually,
sorting is a highly memory intense business.

\section{Do you need to?}

There are some algorithms that seem to require sorted data, but don't, and some
that require sorted data but don't seem to. Be sure you know whether you need
to before you make any false moves.

One common use of sorting in games is in the render pass where some engine
programmers recommend having all your render calls sorted by a high bit count
key generated from a combination of depth, mesh, material, shader, and other
flags such as whether the call is alpha blended. This then allows the renderer
to adjust the sort at runtime to get the most out of the bandwidth available.
In the case of the rendering list sort, you could run the whole list through a
general sorting algorithm, but in reality, there's no reason to sort the alpha
blended objects with the opaque objects, so in many cases you can take a first
step of putting the list into two separate buckets, and save some n for your O.
Also, choose your sorting algorithm wisely. With opaque objects, the most
important part is usually sorting by textures then by depth, but that can
change with how much your fill rate is being trashed by overwriting the same
pixel multiple times. If your overdraw doesn't matter too much but your texture
uploads do, then you probably want to radix sort your calls. With alpha blended
calls, you just have to sort by depth, so choose an algorithm that handles that
case best, usually a quick sort or a merge sort bring about very low but
guaranteed accurate sorting.

\section{Maintain by insertion sort or parallel merge sort}

Depending on what you need the list sorted for, you could sort while modifying.
If the sort is for some AI function that cares about priority, then you may as
well insertion sort as the base heuristic commonly has completely orthogonal
inputs. If the inputs are related, then a post insertion table wide sort might
be in order, but there's little call for a full scale sort.

If you really do need a full sort, then use an algorithm that likes being
parallel. Merge sort and quick sort are somewhat serial in that they end or
start with a single thread doing all the work, but there are variants that work
well with multiple processing threads, and for small data sets there are
special sorting network techniques that can be faster than better algorithms
just because they fit the hardware so well\footnote{Tony Albrecht proves this
point in his article on sorting networks
http://seven-degrees-of-freedom.blogspot.co.uk/2010/07/question-of-sorts.html}.

\section{Sorting for your platfom}

Always remembering that in data-oriented development you must look to the data
before deciding which way you're going to write the code. What does the data
look like? For rendering, there is a large amount of data with different axes
for sorting. If your renderer is sorting by mesh and material, to reduce vertex
and texture uploads, then the data will show that there are a number of render
calls that share texture data, and a number of render calls that share vertex
data. Finding out which way to sort first could be figured out by calculating
the time it takes to upload a texture, how long it takes to upload a mesh, how
many extra uploads are required for each, then calculating the total scene
time, but mostly, profiling is the only way to be sure. If you want to be able
to profile, or allow for runtime changes in case your game has such varying
asset profiles that there is no one solution to fit all, having a flexible
sorting criteria is extremely useful, and sometimes necessary. Fortunately, it
can be made just as quick as any inflexible sorting technique, bar a small set
up cost.

Radix sort is the fastest serial sort. If you can do it, radix sort is very
fast because it generates a list of starting points for data of different
values. This allows the sorter to drop their contents into containers based on
a translation table, a table that returns an offset for a given data value. If
you build a list from a known small value space, then radix sort can operate
very fast to give a coarse first pass. The reason radix sort is serial, is that
it has to modify the table it is reading from in order to update the offsets
for the next element that will be put in the same bucket.

It is possible to make this last stage of the process parallel by having each
sorter ignore any values that it reads that are outside its working set,
meaning that each worker reads through the entire set of values gathering for
their bucket, but there is still a small chance of non-linear performance
due to having to write to nearby memory on different threads.  During
the time the worker collects the elements for its bucket, it could be
generating the counts for the next radix in the sequence, only requiring a
summing before use in the next pass of the data, mitigating the cost of
iterating over the whole set with every worker.

If your data is not simple enough to radix sort, you might be better off using
a merge sort or a quick sort, but there are other sorts that work very well if
you know the length of your sortable buffer at compile time, such as sorting
networks. Through merge-sort is not itself a concurrent algorithm, the many
early merges can be run in parallel, only the final merge is serial, and with a
quick pre-parse of the to-be-merged data, you can finalise with two threads
rather than one by starting from both ends (you need to make sure that the
mergers don't run out of data). Though quick sort is not a concurrent algorithm
each of the sub stages can be run in parallel.  These algorithms are inherently
serial, but can be turned into partially parallelisable algorithms with O(log
n) latency.

When your n is small enough, a traditionally good technique is to write an in
place bubble sort. The algorithm is so simple, it is hard to write wrong, and
because of the small number of swaps required, the time taken to set up a
better sort could be better spent elsewhere. Another argument for rewriting
such trivial code is that inline implementations can be small enough for the
whole of the data and the algorithm to fit in cache\footnote{It might be wise
to have some templated inline sort functions in your own utility header so you
can utilise the benefits of miniaturistion, but don't drop in a bloated
std::sort}.  As the negative impact of the inefficiency of the bubble sort is
negligible over such a small n, it is hardly ever frowned upon to do this.

If you've been developing data-oriented, you'll have a transform that takes a
table of n and produces the sorted version of it. The algorithm doesn't have to
be great to be better than bubble sort, but notice that it doesn't cost any
time to use a better algorithm as the data is usually in the right shape
already. Data-oriented development naturally leads us to reuse good algorithms.

Sorting networks work by implementing the sort in a static manner. They have
input data and run swap if necessary functions on pairs of values of that input
data before outputting the final. The simplest sorting network is two inputs.

\begin{displaymath}
	\xymatrix{
	A \ar[r] & \ar[r] \ar[dr] & \ar[r] & A' \\
	B \ar[r] & \ar[r] \ar[ur] & \ar[r] & B' }
\end{displaymath}

If the values entering are in order, the sorting crossover does nothing. If the
values are out of order, then the sorting cross over causes the values to swap.
This can be implemented as branchless writes:

\begin{verbatim}
a' <= MAX(a,b)
b' <= MIN(a,b)
\end{verbatim}

This is fast on any hardware. The MAX and MIN functions will need different
implementations for each platform and data type, but in general, branch free
code executes faster than code that includes branches.

Introducing more elements:

\begin{displaymath}
	\xymatrix{
	A \ar[r]   & \ar[r] \ar[ddr] & \ar[rrr] &                 &        & \ar[r] \ar[dr] & \ar[rrr] &                &        & A' \\
	B \ar[rrr] &                 &          & \ar[ddr] \ar[r] & \ar[r] & \ar[r] \ar[ur] & \ar[r]   & \ar[r] \ar[dr] & \ar[r] & B' \\
	C \ar[r]   & \ar[r] \ar[uur] & \ar[rrr] &                 &        & \ar[r] \ar[dr] & \ar[r]   & \ar[r] \ar[ur] & \ar[r] & C' \\
	D \ar[rrr] &                 &          & \ar[uur] \ar[r] & \ar[r] & \ar[r] \ar[ur] & \ar[rrr] &                &        & D' }
\end{displaymath}

What you notice here is that the critical path is not long (just three stages
in total), and as these are all branch free functions, the performance is
regular over all data permutations. With such a regular performance profile, we
can use the sort in ways where the variability of sorting time length gets in
the way, such as just-in-time sorting for sub sections of rendering. If we had
radix sorted our renderables, we can network sort any final required ordering
as we can guarantee a consistent timing.

Sorting networks are somewhat like predication, the branch free way of handling
conditional calculations. Because sorting networks use a min / max function,
rather than a conditional swap, they gain the same benefits when it comes to
the actual sorting of individual elements. Given that sorting networks can be
faster than radix sort for certain implementations, it goes without saying that
for some types of calculation, predication, even long chains of it, will be
faster than code that branches to save processing time. Just such an example
exists in the Pitfalls of Object Oriented Design presentation, concluding that
lazy evaluation costs more than the job it tried to avoid. I have no hard
evidence for it yet, but I believe a lot of AI code could benefit the same, in
that it would be wise to gather information even when you are not sure you need
it, as gathering it might be quicker than deciding not to. For example, seeing
if someone is in your field of vision, and is close enough, might be small
enough that it can be done for all AI rather than just the ones that require
it, or those that require it occasionally.
