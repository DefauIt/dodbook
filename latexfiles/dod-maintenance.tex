\chapter{Maintenance and reuse}

When object-oriented design was first promoted, it advertised that it was
easier to modify and extend existing code bases than the more traditional
procedural approach. Though this is not true in practice, partially due to the
language used by games developers and partially because there is an inherent
coupling between objects with respect to their interfaces, it still remains a
consideration of anyone who uses it when entering into another programming
paradigm. Regardless of their level of expertise, an object-oriented programmer
will cite the extensible, encapsulating nature of object-oriented development
as a boon when it comes to working on larger projects.

Highly experienced but more objective developers have admitted or even written
about how C++ is not highly suited to large scale development, but can be used
in it as long as you follow strict guidelines. [LargeScaleC++] But for
those that cannot immediately see the benefit of the data-oriented development
paradigm with respect to maintenance, and evolutionary development, this
chapter covers why it is easier than working with objects.

\section{Debugging}

The prime causes of bugs are the unexpected side effects of a transform, or an
unexpected corner case where a conditional didn't return the correct value. In
object-oriented programming, this can manifest in many ways, from an exception
caused by de-referencing a null, to ignoring the interactions of the player
because the game logic hadn't noticed it was meant to be interactive.

\subsection{Lifetimes}

One of the most common causes of the null dereference is when an
object's lifetime is handled by a separate object to the one manipulating it.
For example, if you are playing a game where the badguys can die, you have to
be careful to update all the objects that are using them whenever the badguy
gets deleted, otherwise you can end up dereferencing invalid memory which can
lead to dereferencing null pointers because the class has destructed.
data-oriented development tends towards this being impossible as the existence
of an entity in a table implies its processability, and if you leave part of an
entity around in a table, you haven't deleted the entity fully. This is a
different kind of bug, but it's not a crash bug, and it's easier to find and
kill as it's just making sure that when an entity is destroyed, all the tables
it can be part of also destroy their elements too.

\subsection{Avoiding pointers}

When looking for data-oriented solutions to programming problems, we often find
that pointers aren't required, and often make the solution harder to scale.
Using pointers where null values are possible implies that each pointer doesn't
only have the value of the object being pointed at, but also implies a boolean
value for whether or not that instance exists. Removing this unnecessary extra
feature can remove bugs, save time, and reduce complexity.

\subsection{Bad State}

Sometimes a bug is more to do with a game not being in the right state.
Debugging then becomes a case of finding out how the game got into its current,
broken state.

When you encapsulate your state, you hide internal changes. This quickly leads
to adding lots of debugging logs. Instead of hiding, data-oriented suggests
keeping data in simple forms, and potentially leaving it around longer than
required can lead to highly simplified transform inspection. If you have a
transform that appears to work, but for one odd case it doesn't, the simplicity
of adding an assert and not deleting the input data reducing the amount of
guesswork and toil required to generate the bug fix. If you keep most of your
transforms as one-way, that is to say they take from one source, and produce or
update another, but even if you run the code multiple times it will still
produce the same results as it would the first time. The transform is
idempotent. This useful property allows you to find a bug, then rewind and
trace through without having to attempt to rebuild the initial state.

One way of keeping your code idempotent is to write your transforms in a single
assignment style. If you operate with multiple transforms but all leading to
predicated join points, you can guarantee yourself some timings, and you can
look back at what caused the final state to turn out like it did without even
rewinding. If your conditions are condition tables, just leave the inputs
around until validity checks have been completed then you have the ability to
go into any live system and check how it arrived at that state. This alone
should reduce any investigation time to a minimum.

\section{Reusability}

A feature commonly cited by the object-oriented developers that seems to be
missing from data-oriented development is reusability. The idea that you won't
be able to take already written libraries of code and use them again, or on
multiple projects, because the design is partially within the implementation.
To be sure, once you start optimising your code to the particular features of a
software project, you do end up with unreusable code. While developing
data-oriented projects, the assumed inability to reuse source code would be
significant, but it is also highly unlikely. The truth is found when
considering the true meaning of reusability.

Reusability is not fundamentally concerned in reusing source files or
libraries. Reusability is the ability to maintain an investment in information.
A wealth of knowledge for the entity that owns the development IP.

Copyright law has made it hard to see what resources have value in reuse, as it
maintains the source as the object of it's discussion rather than the
intellectual property represented by the source. The reason for this is that
ideas cannot be copyrighted, so by maintaining this stance, the copyrighter
keeps hold of this tenuous link to a right to withhold information. Reusability
comes from being aware of the information contained within the medium it is
stored. In our case, it is normally stored as source code, but the information
is not the source code. With Object-Oriented development, the source can be
adapted (adapter pattern) to any project we wish to venture. However, the
source is not the information. The information is the order and existence of
tasks that can and will be performed on the data. Viewing the information this
way leads to an understanding that any reusability that a programming technique
can provide comes down to it's mutability of inputs and outputs. It's
willingness to adapt a set of temporally coupled tasks into a new usage
framework is how you can find out how well it functions reusably.

In object-oriented development, you apply the information inherent in the code
by adapting a class that does the job, or wrapper it, or use an agent. In
data-oriented development, you copy the functions and schema and transform into
and out of the input and output data structures around the time you apply the
information contained in the data-oriented transform.

Even though, at first sight, data-oriented code doesn't appear as resuable on
the outside, the fact is that it maintains the same amount of information in a
simpler form, so it's more reusable as it doesn't carry the baggage of related
data or functions like object-oriented programming, and doesn't require complex
transforms to generate the input and extract from the output like procedural
progamming tends to generate due to the normalising.

Duck typing, not normally available in object-oriented programming due to a
stricter set of rules on how to interface between data, can be implemented with
templates to great effect, turning code that might not be obviously reusable
into a simple strategy, or a sequence of transforms that can be applied to data
or structures of any type, as long as they maintain a naming convention.

The object-oriented C++ idea of reusability is a mixture of information and
architecture. Developing from a data-oriented transform centric viewpoint,
architecture just seems like a lot of fluff code. The only good architecture
that's worth saving is the actualisation of data-flow and transform. There are
situations where an object-oriented module can be used again, but they are few
and far between because of the inherent difficulty interfacing object-oriented
projects with each other.

The most reusable object-oriented code appears as interfaces to agents into a
much more complex system. The best example of an object-oriented approach that
made everything easier to handle, was highly reusable, and was fully
encapsulated was the \texttt{FILE} type from \texttt{stdio.h} that is used as
an agent into whatever the platform and OS would need to open, access, write,
and read to and from a file on the system.

\section{Unit Testing}

Unit testing can be very helpful when developing games, but because of the
object-oriented paradigm making programmers think about code as representations
of objects, and not as data transforms, it's hard to see what can be tested.
Linking together unrelated concepts into the same object and requiring complex
setup state before a test can be carried out, has given unit testing a stunted
start in games as object-oriented programming caused simple tests to be hard to
write. Making tests is further complicated by the addition of the non-obvious
nature of how objects are transformed when they represent entities in a game
world. It can be very hard to write unit tests unless you've been working with
them for a while, and the main point of unit tests is that someone that doesn't
fully grok the system can make changes without falling foul of making things
worse.

Unit testing is mostly useful during refactorings, taking a game or engine from
one code and data layout into another one, ready for future changes. Usually
this is done because the data is in the wrong shape, which in itself is harder
to do if you normalise your data as you're more likely to have left the data in
an unconfigured form. There will obviously be times when even normalised data
is not sufficient, such as when the design of the game changes sufficient to
render the original data-analysis incorrect, or at the very least, ineffective
or inefficient.

Unit testing is simple with data-oriented technique because you are already
concentrating on the transform. Generating tables of test data would be part of
your development, so leaving some in as unit tests would be simple, if not part
of the process of developing the game. Using unit tests to help guide the code
could be considered to be partial following the test-driven development
technique, a proven good way to generate efficient and clear code.

Remember, when you're doing data-oriented development your game is entirely
driven by stateful data and stateless transforms. It is very simple to produce
unit tests for your transforms. You don't even need a framework, just an input
and output table and then a comparison function to check that the transform
produced the right data.

\section{Refactoring}

During refactoring, it's always important to know that you've not broken
anything by changing the code. Allowing for such simple unit testing gets you
halfway there. Another advantage of data-orieted development is that, at every
turn, it peels away the unnecessary elements, so you might find that
refactoring is more a case of switching out the order of transforms more than
changing how things are represented. Refactoring normally involves some new
data representation, but as long as you normalise your data, there's going to
be little need of that. When it is needed, tools for converting from one schema
to another can be written once and used many times.

