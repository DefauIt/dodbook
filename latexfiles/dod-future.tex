\chapter{Future}

The future of hardware is almost certainly increasingly parallel processing
with all the related problems that entails; from power consumption, through
deserialising our processing, to distribution of jobs across more than just
a few local cores.

Supercomputers have been vector machines for decades, and any advances in
graphics cards comes with an increase in the number of cores that run
transforms. Mobile phones have 4 cores and even some microcontrollers have
multi-core layouts\footnote{Parallax Propeller is a really interesting
microcontroller that's been out for some time, but provides a unique
perspective on parallel computing} Stream processing hardware is showing to be
more capable of increasing throughput per generation than the general purpose
computing architecture employed in the design of CPUs and micro-controllers.
It's shaping up that the coming generations of hardware are going to be a
battles of the parallel processing beasts\footnote{Parallela was a successful
2012 Kickstarter campaign for Adapteva, in which their roadmap points to 1024
core boards being produced in 2014}, larger core counts and more numerous
pipes. Memory always seems to be at a premium, so again, we'll probably be
trying to push next gen games out without having enough space in which to work.
One area that has not been pushed around much has been offloading processing
outside the local machine. Even though there has been plenty of work done with
very high latency architectures such as Beowulf clusters or grid computing, the
idea that they could be useful for games is still in its infancy due to the
virtually real-time nature of processing for games. It could be a while off,
but we may see it come one day, and it's better not to let it sneak up on us.
Once we need it, we will need a language that handles offloading to external
processing like any other threaded task. Erlang is one such language. Erlang is
functional (with a tendency for immutable state which helps very much with
concurrency) and is highly process oriented with a very simple migration path
from single-machine multi-threaded to remote processing or grid compute style
distributed processing and a large number of options for fault tolerance.
Node-js offers some of the same parallelism and a much shorter learning time as
it is based on Javascript.  Functional languages would seem to dominate, but
OpenCL isn't purely functional, and C++AMP only requires you consider how to
amplify your code with parallelism, which might not be enough for how many
cores we really end up.  Whatever language we do end up using to leverage the
power of parallel processing, once it's ubiquitous, we'd better be sure we're
not still thinking serial. 

But parallel processing isn't the only thing on the horizon. Along with many
cores comes the beginnings of a new issue that we're only starting to see
traces of outside supercomputing. The issue of power per watt. In mobile
gaming, though we're striving to make a game that works, one thing that
developers aren't regularly doing that will affect sales in the years to come,
is keeping the power consumption down on purpose. The mobile device users will
put up with us eking out the last of the performance for only so long. There
is a bigger thing at stake than being the best graphical performance and
fastest AI code, there is also the need to have our applications be small
enough to be kept on the device, and also not eat up enough battery that the
users drop the application like a hot potato. Data-oriented design can address this
issue, but only if we add it to the list of considerations when developing a
game. As with parallelism, the language we use impacts the possibilities. If we
continue to move towards more high level languages such as C\# and Java, then
we will need smarter compilers to reduce the overhead, but if we develop in a
language made to support tasking, such as the process kernels in OpenCL or the
lightweight threads of Erlang, then we may find new hardware changes to match
the language much the same way C++ and object-oriented design changed the way
CPUs were designed.

\section{Parallel processing}

In short, parallel processing increases the number of things done at the same
time. Towards an infinite number of CPUs the amount of processing available is
infinite, but the amount of processing you can use is limited by two factors.

The first limiting factor is the amount of work that needs to be done. If you
have a perfectly parallel algorithms, then you are limited by the number of
units of work you have to do.

For example, if a graphics card had an infinite number of cores (and we'll keep
using infinite as it is the only good measure for future values of N), then the
maximum number of cores that it can use to render a single polygon would be
measured in how many tasks there are to do. If the polygon was fully covering a
standard 1080P HD screen with 1920 x 1080 pixels, then the highest feasible
number of cores to throw at the task of pixel shading the poly to the back
buffer would be 2,073,600, and thus the highest throughput you can possibly get
from the machine would rely on how fast one single core could transform the
request to render, into a final change of value at the pixel. This example is
slightly broken because the time taken to set up that many cores in a normal
rendering setup would probably cost more time than just splitting the task into
larger chunks, and if you use MSAA you can call on even more cores, but
eventually you'll even run out of samples to do. The point being, if there are
not enough jobs, then the cores will be under utilised.

A lack of jobs is hard to find in a graphics card, which is part of the reason
they have been growing in power so rapidly over the last few years (2008-2013),
and show no signs of significantly slowing down in their GFLOPS growth. There
are always more pixels to render, always more vertices to transform, always
more textures to lookup. The bottleneck in a graphics card is still the number
of cores, and that's why it's relatively easy to increase the power of them
compared to general purpose chips like CPUs.

In addition to the amount of jobs the cores have to do, the relative similarity
of the jobs makes it easier to move towards a job-board style approach to job
dispatch. In every parallel architecture so far, the instructions to run per
compute core are decided specifically per core. In the future, an implicit job
system may make a difference by instigating the concept of a public read-only
job spec, and having set up the compute cores to look out for jobs and apply
their own variant information on them. A good example might be that a graphics
card may set all the compute cores to run the same algorithm, but given some
constants about where they should get their data such as offsets into the
stream and different output rectangles of the display.

This second limiting factor is dependency. We touched on it with the argument
for pixel shaders, the set up time is a dependency. The number of cores able to
be utilised is serial up until the first step of set up is finished, that is,
the call to render primitive. Once the render call is done, we can bifurcate
our way to parallelism, but even then, we're wasting a lot of power because of
dependence on previous steps. One way to avoid this in hardware is have many
cores wait for tasks to do by assuming they will be in charge of some
particular part of the transform. We see it in SIMD processing where the CPU
issues one instruction to multiply a vector, and each sub core carries out its
own multiply, knowing that even though it was told that a multiply for the
whole vector was issued, it could be sure that it was the core involved in
multiplying element n of that vector. This is very useful and possibly why
future optimisations of hardware can work towards a runtime configurable
hardware layout that runs specified computations on demand, but without
explicit instruction. Instead, as soon as any data enters into the
transformation arena, it begins the configured task without asking for, or
needing information on how to interpret the data. This type of stream
processing may be best suited to runtime configurable hardware.

Ahmdal's Law is based on two constants, the time that a process must be serial
and the time that a process can be parallel. In the world of infinite core
computing we must continually strive for the lowest possible serial latency in
our development. That means we must find out what is critical, what can be done
without prior information, what can be done in preparation, what can be not
done until the very last moment. All these elements of processing add up to a
full product, and without considering what the output data is and how we get
there, we could continually return to the state where we are optimising for
code, and not for what is of ultimate importance, the experience due to
realisation of output data in a timely fashion.

\section{Distributed computing}

When you think of distributed computing, normally you think of farms of
computers spread over multiple locations, running long lifetime processes on
massive amounts of data. You think of on demand load balancing systems
providing more compute where necessary on the grid. All this talk of large
scale is just another step out from our CPU cores. Another layer that we can
move to when the possibility presents itself. One day, maybe, we will have
invisible grid computing for the consumer. The idea of spreading the compute
load out from the physical device was first presented commercially with the
CellBE, the idea that adding more compute to an existing hardware instance
could be done without explicitly linking hardware through a proprietary
interface. There were articles talking of how your TV could increase the
quality of your gaming experience by allowing the console to offload some of
the processing onto the TVs processor. This has not come to pass, but the
common core, the idea of doing more work by adding more machines, that has not
gone away and is in common use in most development companies, whether by use of
the free to use build accelerators such as the one included in the Sony
developer tools, or proprietary software such as Xoreax's IncrediBuild. Adding
more compute to processes that can safely be very high latency is definitely
one way of using grid engines, but another alternative is to use grid engine's
to provide answers to questions not yet asked. If you know that a type of
question is common during your processing, then offloading the question process
so it processes all known variants of the question means you can get back the
answer to all the possible questions before it's even asked.

For example, If you know that your scene is going to animate, and you want to
do a ray cast from some entities to other entities, but don't know which, you
can task the grid engine with advancing the animation and doing all the ray
casts. Then, once you are ready, read from the ray casts that you decided that
you did need after whatever calculations you needed to do. That's a lot of
wasted processing power, but it reduces latency. When you're thinking about the
future, it is only sensible to think about latency as the future contains an
infinite number of processing cores.

Using this one example, you can ramp it back to current generation hardware by
offloading processes to begin large scale ray casting, and during the main
thread calculation stages, cull tasks from the thread that is managing the ray
casts. This means that you can get started on tasks, but only waste cycles when
you don't know enough to not.

With very high speed network adapters, very high bandwidth network connections
to the internet, grid computing inside games might become more commonplace than
expected. What gameplay elements this amount of processing power can open up is
beyond our vision, but once it is here, we can expect remote processing through
something akin to stored procedures to become a staple part of the game
developer tool-kit.

\section{Runtime hardware configuration}

We cannot know what the future brings, and some assumptions we have about CPUs
might be broken. One such assumption might be that our hardware is fixed in one
form once it has been fabricated. Both field programmable gate arrays and
complex programmable logic arrays allow for change during their lifetime, and
some announcements have been made about mainstream hardware adding FPGAs to
their arsenal\footnote{One really good reason for an FPGA is to allow for
future proofing of output connectivity. FPGAs can be used to test out new
standards such as a new HDMI specification or a completely different encoding}
which may open the door to FPGA add-on cards much the same way we saw the
take-off of graphics cards once they got a foot in the door.

FPGAs present a new and interesting problem to programmers, specifically to the
programmers that may be inventing new plans for others to consume. One way this
could pan out is by re-orienting the mindset of the average programmer into
that of a flow based or stream processing developer. Being able to take game
data and manipulate it with highly efficient, and low latency modules
dynamically loaded onto FPGAs could be the final nail in the coffin for any
language that links code with data.

Approaches to development similar to that imposed on us by the shader model and
the process of getting data into the right layout for the shaders might have
been good training, readying us for the coming days of compute power only
really being available if we order it for later. With dynamic partial
reconfiguration, we would have the same benefits seen in being able to switch
shaders mid render. That is, the ability to utilise an FPGA much smaller than
would be necessary if we were to attempt to cram all the potential processing
modules onto it at once. We've been here before. This extremely high latency
before being able to compute, followed by very fast processing once the
computational framework is ready, is very similar to programming with shaders,
so much so that we might not take that long bringing our existing engines up to
speed.

The CellBE was an attempt at this way of working, but was overlooked by many as
just a strange piece of hardware. The core, an underpowered CPU that would look
after feeding all the high power SPUs was assumed to be a general purpose CPU.
This was an unfortunate side effect of too many years developing for
out-of-order CPUs, and a deep investment in random memory access programming
methods such as object-oriented design and interpreted languages. We don't
know what other CPUs may come along in the future, but we can attempt to use a
data-oriented approach and not try to make a CPU work the way we want to work,
but work with it to make the best out of what we have.

\section{Data centric development in hardware design}

Once software solutions concentrate on transforming data, what changes can we
expect from hardware vendors? Is that a silly question? Would a change of
programming paradigm really affect the people who create hardware?

What should we promote and demote in order to get the most
from our transistors?

move away from serial thinking
more functional programming, where there are no side-effects, leads to
solutions for infinite numbers of cores

Whatever the future brings with respect to hardware and processing
configurations, there are certain assumptions we can make. 
