\chapter{Component Based Objects}

Component oriented design is a good head start for high level data-oriented
design. Components put your mind in the right frame for not linking together
concepts when they don't need to be, and component based objects can easily be
processed by type, not by instance, which leads to a smoother processing
profile. Component based entity systems are found in games development as a
way to provide a data-driven functionality system for entities, which can allow
for designer control over what would normally be the realm of a programmer. Not
only are component based entities better for rapid design changes, but they
also stymie the chances of the components getting bogged down into monolithic
objects as most game designers would demand more components with new features
over extending the scope of existing components.  Most new designs need
iterating, and extending an existing component by code doesn't allow design to
change back and forth, trying different things.

When people talk about how to create compound objects, sometimes they talk
about using components to make up an object. Though this is better than a
monolithic class, it is not component based approach, it merely uses components
to make the object more readable, and potentially more reusable. Component
based turns the idea of how you define an object on its head. The normal
approach to defining an object in object oriented design is to name it, then
fill out the details as and when they become necessary. For example, your car
object is defined as a Car, if not extending Vehicle, then at least including
some data about what physics and meshes are needed, with construction arguments
for wheels and body shell model assets etc, possibly changing class dependent
on whether it's an AI or player car. In component oriented design, component
based objects aren't so rigidly defined, and don't so much become defined after
they are named, as much as a definition is selected or compiled, and then
tagged with a name if necessary. For example, instancing a physics component
with four wheel physics, instancing a renderable for each part (wheels, shell,
suspension) adding an AI or player component to control the inputs for the
physics component, all adds up to something which we can tag as a Car, or leave
as is and it becomes something implicit rather than explicit and immutable.

A component based object is nothing more than the sum of its parts. This means
that the definition of a component based object is also nothing more than an
inventory with some construction arguments.  This object or definition agnostic
approach makes refactoring and redesigning a completely trivial exercise.

\section{Components in the wild}

Component based approaches to development have been tried and tested. Many high
profile studios have used component driven entity systems to great
success\footnote{Gas Powered Games, Looking Glass Studios, Insomniac, Neversoft
all used component based objects}, and this was in part due to their developer's
unspoken understanding that objects aren't a good place to store all your data
and traits. Gas Powered Games' Dungeon Siege Architecture is probably the
earliest published document about a game company using a component based
approach. If you get a chance, you should read the article\footnote{Gas Powered
Games released a web article about their component based architecture back in
2004, read it at http://garage.gaspowered.com/?q=su\_301}. In the article it
explains that using components means that the entity type\footnote{GPG:DG uses
GO or Game-Objects, but we stick with the term entity because it has become the
standard term.} doesn't need to have the ability to do anything. Instead, all
the attributes and functionality come from the components of which the entity
is made.

In this section we'll show how we can take an existing class and rewrite it in
a component based fashion. We're going to tackle a fairly typical complex
object, the player class. Normally these classes get messy and out of hand
quite quickly, and we're going to assume it's a player class designed for a
generic 3rd person action game, and take a typically messy class as our
starting point.

\begin{lstlisting}[caption=Player class]
class Player {
public:
	Player();
	~Player();
	Vec GetPos(); // the root node position
	void SetPos( Vec ); // for spawning
	Vec GetSpeed(); // current velocity
	float GetHealth();
	bool IsDead();
	int GetPadIndex(); // the player pad controlling me
	float GetAngle(); // the direction the player is pointing 
	void SetAnimGoal( ... ); // push state to anim-tree
	void Shoot( Vec target ); // fire the player's weapon
	void TakeDamage( ... ); // take some health off, maybe animate for the damage reaction
	void Speak( ... ); // cause the player to start audio/anim
	void SetControllable( bool ); // no control in cut-scene
	void SetVisible( bool ); // hide when loading / streaming
	void SetModel( ... ); // init streaming the meshes etc
	bool IsReadyForRender();
	void Render(); // put this in the render queue
	bool IsControllable(); // player can move about?
	bool IsAiming(); // in normal move-mode, or aim-mode
	bool IsClimbing();
	bool InWater(); // if the root bone is underwater
	bool IsFalling();
	void SetBulletCount( int ); // reload is -1
	void AddItem( ... ); // inventory items
	void UseItem( ... );
	bool HaveItem( ... );
	void AddXP( int ); // not really XP, but used to indicate when we let the player power-up
	int GetLevel(); // not really level, power-up count
	int GetNumPowerups(); // how many we've used
	float GetPlayerSpeed(); // how fast the player can go
	float GetJumpHeight();
	float GetStrength(); // for melee attacks and climb speed
	float GetDodge(); // avoiding bullets
	bool IsInBounds( Bound ); // in trigger zone?
	void SetGodMode( bool ); // cheater
private:
	Vec pos, up, forward, right;
	Vec velocity;
	Array<ItemType> inventory;
	float health;
	int controller;
	AnimID currentAnimGoal;
	AnimID currentAnim;
	int bulletCount;
	SoundHandle playingSoundHandle; // null most of the time
	bool controllable;
	bool visible;
	AssetID playerModel;
	LocomotionType currentLocomotiveModel;
	int xp;
	int usedPowerups;
	int SPEED, JUMP, STRENGTH, DODGE;
	bool cheating;
};
\end{lstlisting}

\section{Away from the hierarchy}

A recurring theme in articles and post-mortems from people moving from
Object-Oriented hierarchies of gameplay classes to a component based approach
is the transitional states of turning their classes into containers of smaller
objects, an approach often called composition. This transitional form takes an
existing class and finds the boundaries between the concepts internal to the
class and refactors them out into new classes that can be pointed to by the
original class.

First, we move the data out into separate structures so they can be more easily
combined into new classes.

\begin{lstlisting}
struct PlayerPhysical {
	Vec pos, up, forward, right;
	Vec velocity;
};
struct PlayerGameplay {
	float health;
	int xp;
	int usedPowerups;
	int SPEED, JUMP, STRENGTH, DODGE;
	bool cheating;
};
struct EntityAnim {
	AnimID currentAnimGoal;
	AnimID currentAnim;
	SoundHandle playingSoundHandle; // null most of the time
};
struct PlayerControl {
	int controller;
	bool controllable;
};
struct EntityRender {
	bool visible;
	AssetID playerModel;
};
struct EntityInWorld {
	LocomotionType currentLocomotiveModel;
};
struct Inventory {
	Array<ItemType> inventory;
	int bulletCount;
};

SparseArray<PlayerPhysical> phsyicalArray;
SparseArray<PlayerGameplay> gameplayArray;
SparseArray<EntityAnim> animArray;
SparseArray<PlayerControl> controlArray;
SparseArray<EntityRender> renderArray;
SparseArray<EntityInWorld> inWorldArray;
SparseArray<Inventory> inventoryArray;

class Player {
public:
	Player();
	~Player();
	// ...
	// ... the member functions
	// ...
private:
	int EntityID;
};
\end{lstlisting}

When we do this, we see how a first pass of building a class out of smaller
classes can help organise the data into distinct purpose oriented collections,
but we can also see the reason why a class ends up being a tangled mess. The
rendering functions need access to the player's position as well as the model,
and the gameplay functions such as Shoot need access to the inventory as well
as setting animations and dealing damage. Take damage will need access to the
animations, health. Things are already seeming more difficult to handle than
expected. But what's really happening here is that you can see that code needs
to cut across different pieces of data. With this first pass, we can start to
see that functionality and data don't belong together.

We move away from the class being a container for the classes immediately as we
can quickly see the benefit of cache locality when we're iterating over
multiple entities doing related tasks. If we were iterating over the entities
checking for whether they were out of bullets, and if so setting their
animation to reloading, we would only need to check the EntityID to find the
element for each involved entity.

Functionality of a class, or an object, comes from how facts are manipulated.
The relations between the facts are part of the problem domain, but the facts
are only raw data. This separation of fact from meaning is not possible with an
object oriented approach, which is why every time a fact acquires a new
meaning, the meaning has to be implemented as part of the class containing the
fact. Removing the facts from the class and instead keeping them as separate
components has given us the chance to move away from classes that have meaning,
but at the expense of having to look up facts from different sources.

\section{Towards Managers}

After splitting your classes up into components, you might find that your
classes look more awkward now they are accessing variables hidden away in new
structures. But it's not your classes that should be looking up variables, but
instead transforms on the classes. A common operation such as rendering
requires the position and the model information, but it also requires access to
the renderer. Such global access is seen as a common compromise during most
game development, but here it can be seen as the method by which we move away
from a class centric approach to transforming our data into render requests
that affect the graphics pipeline without referring to data unimportant to the
renderer, and how we don't need a controller to fire a shot.

\begin{lstlisting}
class RenderManager {
	void Update() {
		foreach( {index,pos} in positionArray ) {
			if( index in renderArray ) {
				ModelID mid = renderArray[ index ].playerModel;
				gRenderer.AddModel( mid, pos );
			}
		}
	}
}
class PhysicsManager {
	void Update() {
		foreach( {index,pos} in positionArray ) {
			if( index in renderArray ) {
				ModelID mid = renderArray[ index ].playerModel;
				ApplyCollisionAndResponse( pos.position, pos.velocity
			}
		}
	}
}
class PlayerControl {
	void Update() {
		foreach( {index,control} in controlArray ) {
			if( control.controllable ) {
				Pad pad = GetPad( control.controller );
				if( pad.IsPressed( SHOOT ) ) {
					if( index in inventoryArray ) {
						if( inventoryArray[ index ].bulletCount > 0 ) {
							if( index in animArray ) {
								animArray[ index ].currentAnimGoal = SHOOT_ONCE;
							}
						}
					}
				}
			}
		}
		foreach( {index,inv} in inventoryArray )  {
			if( inv.bulletCount > 0 ) {
				if( index in animArray ) {
					anim = animArray[ index ];
					if( anim.currentAnim == SHOOT_ONCE ) {
						anim.currentAnim = SHOT;
						inventoryArray[index].bulletCount -= 1;
						anim.playingSoundHandle = PlaySound( GUNFIRE );
					}
				}
			}
		}
	}
}
\end{lstlisting}

What happens when we let more than just the player use these arrays? Normally
we'd have some separate logic for handling player fire until we refactored the
weapons to be generic weapons with NPCs using the same code for weapons proably
by having a new weapon class that can be pointed to by the player or an NPC,
but instead what we have here is a way to split off the weapon firing code in
such a way as to allow the player and the NPC to share firing code without
inventing a new class to hold the firing. In fact, what we've done is split the
firing up into the different tasks that it really contains.

Tasks are good for parallel processing, and with component based objects we
open up the opportunity to make most of our previously class oriented processes
into more generic tasks that can be dished out to whatever CPU or co-processor
can handle them.

\section{There is no Entity}

What happens when we completely remove the player class? If we consider that an
entity may be not only represented by it's collection of components, but also
might only be it's current configuration of components, then there is the
possibility of removing the core class of Player.  Removing this class can mean
we no longer think of the player as being the centre of the game, but also the
class no longer existing means that any code is no longer tied to itself.

\begin{lstlisting}
struct Orientation { Vec pos, up, forward, right; };
SparseArray<Orientation> orientationArray;
SparseArray<Vec> velocityArray;
SparseArray<float> healthArray;
SparseArray<int> xpArray, usedPowerupsArray, controllerID, bulletCount;
struct Attributes { int SPEED, JUMP, STRENGTH, DODGE; };
SparseArray<Attributes> attributeArray;
SparseArray<bool> godmodeArray, controllable, isVisible;
SparseArray<AnimID> currentAnim, animGoal;
SparseArray<SoundHandle> playingSound;
SparseArray<AssetID> modelArray;
SparseArray<LocomotionType> locoModelArray;
SparseArray<Array<ItemType> > inventoryArray;

void RenderUpdate() {
	foreach( {index,assetID} in modelArray ) {
		if( index in isVisible ) {
			gRenderer.AddModel( assetID, orientationArray[ index ] );
		}
	}
}
void PhysicsUpdate {
	foreach( {index,vel} in velocityArray ) {
		if( index in modelArray ) {
			ApplyCollisionAndResponse( orientationArray[index], vel, modelArray[ index ] );
		}
	}
}
void ControlUpdate() {
	foreach( {index,control} in controllerID ) {
		if( controllable[ index ] ) {
			Pad pad = GetPad( control.controller );
			if( pad.IsPressed( SHOOT ) ) {
				if( inventoryArray[ index ].bulletCount > 0 ) {
					animArray[ index ].currentAnimGoal = SHOOT_ONCE;
				}
			}
		}
	}
}
void UpdateAnims() {
	foreach( {index, anim} in currentAnim ) {
		if( anim == SHOOT_ONCE ) {
			anim = SHOT;
			inventoryArray[index].bulletCount -= 1;
			playingSound = PlaySound( GUNFIRE );
		}
	}
}
int NewPlayer( int padID, Vec startPoint ) {
	int ID = newID();
	controllerID[ ID ] padID;
	GetAsset( "PlayerModel", ID ); // adds a request to put the player model into modelArray[ID]
	orientationArray[ ID ] = Orientation(startPoint);
	velocityArray[ ID ] = VecZero();
	return ID;
}
\end{lstlisting}

Moving away from a player class means that many other classes can be invented
without adding much code. Allowing script to generate new classes by
composition increases the power of script to dramatically increase the apparent
complexity of the game without adding more code. What is also of major benefit
is that all the different entities in the game now run the same code at the
same time. everyone's physics is updated before they are rendered\footnote{or
is updated while the rendering is happening on another thread} and everyone's
controls (whether they be player or AI) are updated before they are animated.
Having the managers control when the code is executed is a large part of the
leap towards fully parallelisable code.

