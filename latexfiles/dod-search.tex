\chapter{Searching}

When looking for specific data, it's very important to remember why you're
doing it. If the search is not necessary, then that's your biggest possible
saving. Finding if a row exists in a table will be slow if approached naively.
You can manually add searching helpers such as binary trees, hash tables, or
just keep your table sorted by using ordered insertion whenever you add to the
table. If you're looking to do the latter, this could slow things down, as
ordered inserts aren't normally concurrent, and adding extra helpers is
normally a manual task. In this section we find ways to combat all these
problems.

\section{Indexes}

Database management systems have long held the concept of indexes. They are
traditionally automatically added when a database management system notices
that a particular query has been run a large number of times. We can use this
idea and implement a just-in-time indexing system in our games to provide the
same kinds of performance improvement. Every time you want to find out if an
element exists, or even just generate a subset, like when you need to find all
the entities in a certain range, you will have to build it as a pocess to
generate a row or table, but the process that causes the row or table
generation can be thought of as an object that can transform itself dependent
on the situation. Starting out as a simple linear search query (if the data is
not already sorted), the process can find out that it's being used quite often
through internal telemetry, and proably be able to discover that it's generally
returns a simply tunable set of results, such as the first N items in a sorted
list, and can hook itself into the tables that it references. Hooking into the
insertion, modification, and deletion would allow the query to update itself
rather than run the full query again. This can be a significant saving, but it
can also be useful as it is optional. As it's optional, it can be garbage
collected.

If we build generalised back ends to handle building queries into these tables,
they can provide multiple benefits. Not only can we provide garbage collection
of indexes that aren't in use, but they can also make the programs in some way
self documenting and self profiling. If we study the logs of what tables had
pushed for building indexes for their queries, then we can quickly see data
hotspots and where there is room for improvement. The holy grail here is self
optimising code.

\section{data-oriented Lookup}

The first step in any data-oriented approach to searching is to understand the
difference between the key, and the data that is dependent on the key.
Object-oriented solutions to searching often ask the object whether or not it
satisfies some criteria. Because the object is asked a question, there is often
a large amount of code called, memory indirectly accessed, and cache-lines
filled but hardly used. Even in a non-Object-oriented approach, there is
frequently still a lot of cache abuse. Following is a listing for a simple
binary search for a key in a naive implementation of an animation container.
This kind of data access pattern is common in animation, hash lookups, and other
spatial searches.

\begin{lstlisting}[caption=Binary search for a given time value]
struct AnimKey {
	float time;
	float x,y,z;
};
struct Anim {
	int numKeys;
	AnimKey *keys;
	AnimKey *GetKeyAtTime( float t ) {
		int l = 0, h = numKeys-1;
		int m = (l+h) / 2;
		while( l < h ) {
			if( keys[m].time > t )
				h = m;
			else
				l = m;
			m = (l+h) / 2;
		}
		return keys[i];
	}
};
\end{lstlisting}

We can improve on this very quickly by moving to a structure of arrays. What
follows is a quick rewrite that saves us a lot of memory access.

\begin{lstlisting}[caption=Binary search for a given time value]
struct AnimKey {
	float x,y,z;
};
struct Anim {
	int numKeys;
	float *keyTime;
	AnimKey *keys;
	AnimKey *GetKeyAtTime( float t ) {
		int l = 0, h = numKeys-1;
		int m = (l+h) / 2;
		while( l < h ) {
			if( keyTime[m] > t )
				h = m;
			else
				l = m;
			m = (l+h) / 2;
		}
		return keys[i];
	}
};
\end{lstlisting}

Using the data-oriented SoA, we speed this up, but why is it faster? The first
impression most people get is that we've moved from a nice solid 16byte
structure to a horrible or padded 12byte structure for the animation keys.
Sometimes it pays to think a bit further than what looks right as first glance.
Primarily, we are now finding the key by hunting for a value in a list of
values. Not looking for a whole struct by one of its members in an array of
structs. This is faster because the cache will be filled with mostly pertinent
data. There are ways to organise the data better still, but any more
optimisation requires a complexity or space time trade off. A basic binary
search will home in on the correct data quite quickly, but each of the first
steps will cause a new cache-line to be read in.  If you know how big your
cache-line is, then you can check all the values that have been loaded for free
while you wait for the next cache-line to load in.  Once you have got near the
destination, most of the memory you need is in memory and all you're doing
from then on is making sure you have found the right key.  In a good SoA
engine, all this can be done behind the scenes with a well optimised search
algorithm usable all over the game code. It is worth mentioning again, that
every time you break out into larger data structures, you deny your proven
code the chance to be reused.

If the reason for a search is simpler, such as checking for existence, then
there are even faster alternatives. Bloom filters offer a constant time lookup
that even though it produces some false positives, can be tweaked to generate a
reasonable answer hit rate for very large sets. In particular, if you are
checking for which table an row exists in, then bloom filters work very well,
by providing data about which tables to look in, usually only returning the
correct table, but sometimes more than one.

In relational databases, indexes are added to tables at runtime when there are
multiple queries that could benefit in their presence. For our data-oriented
approach, there will always be some way to speed up a search but only by
looking at the data. If the data is not already sorted, then an index is a
simple way to find the specific item we need. If the data is already sorted,
but needs even faster access, then a search tree optimised for the cache-line
size would help.

A binary search is one of the best search algorithms for using the smallest
number of instructions to find a key value. But if you want the fastest
algorithm, you must look at what takes time, and a lot of the time, it's not
instructions. Loading a whole cache-line or information and doing as much as
you can with that would be a lot more helpful than using the smallest number of
instructions. Consider these two different data layouts for a animation key
finding system:

\begin{lstlisting}[caption=Anim key formats]
struct Anim {
	float keys[num];
	VALUE_TYPE values[num];
};

struct KeyClump { float startTimes[32]; } // 128 bytes
keyClump root;
keyClump subs[32];
// implicit first element that has the same value as root.startTimes[n],
// but last element is the same as root.startTimes[n+1]
VALUE_TYPE Values[32*32];
\end{lstlisting}

A binary search is quick, but if you know you are going to be doing completely
random access of your data, but your data is of a certain size, you can make
even more optimisations. In the key clumps version, the search loads the root
(which has the times of every 32\textsuperscript{nd} key,) then after finding
which subroot to load loads that subroot. Once these two pieces of data are
memory, there is enough information to generate a lookup index for the two
values stored in {\em Values}, and the sub key interval interpolation value.

\begin{lstlisting}[caption=Finding in the clump]
VALUE_TYPE GetAtT( t ) 
	int rootIndex = binaryFind( t, root );
	KeyClump &theSub = subs[rootIndex];
	int subIndex = binaryFind( t, theSub );
	int keyIndex = rootIndex * 32 + subIndex;
	float *keyTimes = &theSub;
	const float t1 = keyTimes[subIndex];
	const float t2 = keyTimes[subIndex+1];
	float blend = (t-t1) / ( t2-t1 );
	return Interp( Values[keyIndex], Values[keyIndex+1], blend );
}
\end{lstlisting}

But most data isn't this simple to optimise. A lot of the time, we have to work
with spatial data, but because we use objects, it's hard to strap on an
efficient spatial reference container after the fact. It's virtually impossible
to add one at runtime to an externally defined class of objects.

Adding spatial partitioning when your data is in a simple data format like rows
allows us to generate spatial containers or lookup systems that will be easy to
maintain and optimise for each particular situation. Because of the inherent
reusability in data-oriented transforms, we can write some very highly
optimised routines for the high level programmers.

\section[Finding low and high]{Finding lowest or highest is a sorting problem}

In some circumstances you don't even really need to search. If the reason for
searching is to find something within a range, such as finding the closest
food, or shelter, or cover, then the problem isn't really one of searching, but
one of sorting. In the first few runs of a query, the search might literally do
a real search to find the results, but if it's run often enough, there is no
reason not to promote the query to a runtime updating sorted subset of some
other tables' data. If you need the nearest three elements, then you keep a
sorted list of the nearest three elements, and when an element has an update,
insertion or deletion, you can update the sorted three with that information.
For insertions or modifications that bring elements that are not in the set
closer, you can check that the element is closer and pop the lowest before
adding the new element to the sorted best. If there is a deletion, or a
modification that makes one in the sorted set a contender for elimination, a
quick check of the rest of the elements to find a new best set might be
necessary. If you keep a larger than necessary set of best values, however,
then you might find that this never happens.

\begin{lstlisting}[caption=keeping more than you need]
Array<int> bigArray;
Array<int> bestValue;
const int LIMIT = 3;

void AddValue( int newValue ) {
	bigArray.push( newValue );
	bestValue.sortedinsert( newValue );
	if( bestValue.size() > LIMIT )
		bestValue.erase(bestValue.begin());
}
void RemoveValue( int deletedValue ) {
	bigArray.remove( deletedValue );
	bestValue.remove( deletedValue );
}
int GetBestValue() {
	if( bestValue.size() ) {
		return bestValue.top();
	} else {
		int best = bigArray.findbest();
		bestvalue.push( best );
		return best;
	}
}
\end{lstlisting}

The trick is to find, at runtime, the best value to use that covers the
solution requirement. The only way to do that is to check the data at runtime.
For this, either keep logs, or run the tests with dynamic resizing based on
feedback from the table's query optimiser.

\section[Finding random]{Finding random is a hash/tree issue}

For some tables, the values change very often. For a tree representation to be
high performance, it's best not to have a high number of modifications as each
one could trigger the need for a rebalance. Of course, if you do all your
modifications in one stage of processing, then rebalance, and then all your
reads in another, then you're probably going to be okay still using a tree.  If
you have many different queries on some data, you can end up with
multiple different indexes. How frequently the entries are changed should
influence how you store your index data. Keeping a tree around for each query
could become expensive, but would be cheaper than a hash table in most
implementations. Hash tables become cheaper where there are many modifications
interspersed with lookups, trees are cheaper where the data is mostly static,
or at least hangs around for a while. When the data becomes constant, a perfect
hash can trump a tree. Trees can be quickly made from data, faster than hash
tables which need you to run a hash function per entry being inserted. Hash
tables can be quickly updated and can take up only marginally more space than
trees if implemented with care. Perfect hash tables use precalculated hash
functions to generate an index, and don't require any space other than is used
to store the constants and the array of pointers or offsets into the original
table. If you have the time, then you might find a perfect hash that returns
the actual indexes. I'm not sure we have that long though.

For example, if we need to find the position of someone given their gamertag,
the players won't be sorted by gamertag normally, so we need a gamertag to
player lookup. This data is mostly constant during game so would be better
suited to a tree or perfect hash if we could generate one fast enough.
Generally, people leave games and join them, so a tree is probably as static a
container as is sensible to use. Given the number of players isn't likely to go
over a ten thousand at any point, a trie based off the gamertag hash value
should provide very quick lookup.

A hash can be more useful when the table is more chaotic, for example, messages
have senders and recipients. If we need to do something with all the messages
from or to one sender or recipient, then at least one of those is not going to
be the normal sorted order of the table. Because the table is in constant flux,
keeping a tree is virtually useless, also the hash table variant can handle
multiple entries with the same key a lot easier than a tree implementation as
it can use the standard linked list of entries in a hash bucket to collect all
the messages with a certain sender/recipient.


