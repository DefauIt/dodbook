\chapter{Data-Oriented Design}

Data-oriented design has been around for decades in one form or another, but
was only officially given a name by Noel Llopis in his September 2009 article
of the same name. The idea that it is a programming paradigm is seen as
contentious as many believe that it can be used side by side with another
paradigm such as object-oriented programming, procedural programming, or
functional programming. In one respect they are right, data-oriented design can
function alongside the other paradigms, but so can they. A Lisp programmer
knows that functional programming can coexist with object-oriented programming
and a C programmer is well aware that object-oriented programming can coexist
with procedural programming. We shall ignore these comments and claim that
data-oriented design is another important tool; a tool just as capable of
coexistence as the rest.

The time was right in 2009. The hardware was ripe for a change in how to
develop.  Badly programmed potentially very fast computers, hindered by a
hardware ignorant programming paradigm made many engine programmers weep. The
times have changed, and many mobile and desktop solutions now need the
data-oriented design approach less, not because the machines are better at
mitigating ineffective programming, but the games being designed are less
demanding. As we move towards ubiquity for multi-core machines, not just the
desktops, but our phones too, and towards a world of programming for massively
parallel processing design as in the compute cores on our graphics cards and in
our consoles, the need for data-oriented design will only grow. It will grow
because abstractions and serial thinking will be the bottleneck of your
competitors, and those that embrace the data-oriented approach will thrive.

\section{It's all about the data}

Data is all we have. Data is what we need to transform in order to create a
user experience. Data is what we load when we open a document. Data is the
graphics on the screen and the pulses from the buttons on your game pad and the
cause of your speakers and headphones producing waves in the air and the method
by which you level up and how the bad guy knew where you were to shoot at you
and how long the dynamite took to explode and how many rings you dropped when
you fell on the spikes and the current velocity of every particle in the
beautiful scene that ended the game, that was loaded off the disc and into your
life. Any application is nothing without its data. Photoshop without the images
is nothing. Word is nothing without the characters. Cubase is worthless without
the events.  All the applications that have ever been written have been written
to output data based on some input data. The form of that data can be extremely
complex, or so simple it requires no documentation at all, but all applications
produce and need data.

Instructions are data too. Instructions take up memory, use up bandwidth, and
can be transformed, loaded, saved and constructed. It's natural for a developer
to not think of instructions as being data\footnote{unless they are a Lisp
programmer}, but there is very little differentiating them on older, less
protective hardware.  Even though memory set aside for executables is protected
from harm and modification on most contemporary hardware, this relatively new
invention is still merely an invention. Instructions are still data, and once
more, they are what we transform too. We take instructions and turn them into
actions. The number, size, and frequency of them is something that matters and
that we have control over.

This forms the basis of the argument for a data-oriented approach to
development, but leaves out one major element. All this data and the
transforming of data, from strings, to images, to instructions, they all have
to run on something. Sometimes that thing is quite abstract, such as a virtual
machine running on unknown hardware. Sometimes that thing is concrete, such as
knowing which specific CPU and what speed and memory capacity and bandwidth you
have available. But in all cases, the data is not just data, but data that
exists on some hardware somewhere, and it has to be transformed by some
hardware. In essence, data-oriented design is the practice of designing
software by developing transforms for well formed data where well formed is
guided by the target hardware and the transforms that will operate on it.
Sometimes the data isn't well defined, and sometimes the hardware is equally
evasive, but in most cases a good background of hardware appreciation can help
out almost every software project.

If the ultimate result of an application is data, and all input can be
represented by data, and it is recognised that all data transforms are not
performed in a vacuum, then a software development methodology can be founded
on these principles, the principles of understanding the data, and how to
transform it given some knowledge of how a machine will do what it needs to do
with data of this quantity, frequency, and it's statistical qualities. Given
this basis, we can build up a set of founding statements about what makes a
methodology data-oriented.

\section{Data is not the problem domain}

The first principle: Data is not the problem domain.

For some, it would seem that data-oriented design is the antithesis of
most other programming paradigms because data-oriented design is a technique that does
not readily allow the problem domain to enter into the software so readily. It does
not recognise the concept of an object in any way, as data is consistently
without meaning, whereas the abstraction heavy paradigms try to pretend the
computer and its data do not exist at every turn, abstracting away the idea
that there are bytes, or CPU pipelines, or other hardware features.

The data-oriented design approach doesn't build the real world problem into the
code. This could be seen as a failing of the data-oriented approach by veteran
object-oriented developers, as many examples of the success of object-oriented
design come from being able to bring the human concepts to the machine, then in
this middle ground, a solution can be written in this language that is
understandable by both human and computer. The data-oriented approach gives up
some of the human readability by leaving the problem domain in the design
document, but stops the machine from having to handle human concepts at any
level by just that same action.

Let us consider first how the problem domain becomes part of the software in
programming paradigms that promote abstraction. In the case of objects, we tie
meanings to data by associating them with their containing classes and their
associated functions. In high level abstraction, we separate actions and data
by high level concepts, which might not apply at the low level, thus reducing
the likelihood that the functions will be efficiently implemented.

When a class owns some data, it gives that data a context, and that context can
sometimes limit the ability to reuse the data.  Adding functions to a context
can bring in further data, which quickly leads to classes that contain many
different pieces of data that are unrelated in themselves, but need to be in
the same class because an operation required a context, but the operation also
required additional data. Normally this becomes hard to untangle as functions
that operate over the whole context drag in random pieces of data from all over
the class meaning that many data items cannot be removed as they would then be
inaccessible.

When we consider the data from the data-oriented design point of view, data is
mere facts that can be interpreted in whatever way necessary to get the output
data in the format it needs to be. We only care about what transforms we do,
and where the data ends up. In practice, when you discard meanings from data,
you also reduce the chance of tangling the facts with their contexts, and thus
you also reduce the likelihood of mixing unrelated data just for the sake of an
operation or two.

\section{Data and Statistics}

The second principle: Data is type, frequency, quantity, shape and probability.

The second statement is that data is not just the structure. A common
misconception about data-oriented design is that it's all about cache misses.
Even if it was all about making sure you never missed the cache, and it was all
about structuring your classes so the hot and cold data was split apart, it
would be a generally useful addition to your toolkit of thought, but
data-oriented design is about all aspects of the data. To write a book on how
to avoid cache misses, you need more than just some tips on how to organise
your structures, you need a grounding in what is really happening inside your
computer when it is running your program. Teaching that in a book is also
impossible as it would only apply to one generation of hardware, and one
generation of programming languages, however data oriented design is not rooted
in one language and one set of hardware. The schema of the data is still
important, but the actual values are as important, if not more so. It is not
enough to have some photographs of a cheetah to determine how fast it can run.
You need to see it in the wild.

The data-oriented design model is centred around data, live data, real data,
information data. Object-oriented design is centred around the problem and its
solution. Objects, not real things, but abstract representations of things that
make up the design of the solution to the problem presented in the application
design document. The objects only manipulate the data needed to represent them
without any consideration for the hardware or the real world data patterns or
quantities. This is why object-oriented design allows you to quickly build up
first versions of applications, allowing you to put the first version of the
design document or problem definition directly into the code. 

Data-oriented design takes a different approach to the problem, instead of
assuming that we know nothing about the hardware, it assumes we know nothing
about the problem. Anyone who has written a sizable piece of software should
recognise that the technical design for any project can change so much that
there is hardly anything recognisable from the first draft in the final
implementation. Data-oriented design avoids this waste of resources by never
assuming that the design needs to exist anywhere other than in a document while
it proceeds to provide a solution to the current problem.

Data-Oriented Design takes it's cues from the data that is seen or expected.
Instead of planning for all eventualities, or planning to make things
adaptable, it uses the most probable input to direct the choice of algorithm.
Instead of planning to be extendible, it plans to be simple, and get the job
done. Extendible can be added later, with the safety net of unit tests to
ensure that it remains working as it did while it was simple.

\section{Data changes}

Data-oriented design is current. Object-oriented design starts to show its
weaknesses when designs change. Object-oriented design suffers from an inertia
inherent in keeping the problem domain coupled with the implementation. A
data-oriented approach to design takes note of the change in design by
understanding the change in the data. The data-oriented approach to design also
allows for change to the code when the source of data changes, unlike the
encapsulated internal state manipulations of the object-oriented approach. In
general, data-oriented design handles change better as pieces of data and
transforms can be more simply coupled and decoupled than objects can be mutated
and reused.

The reason this is so comes from the linking of intention and aspect that comes
with lumping data and functions in with concepts of objects where the objects
are the schema, the aspect, the use case, and the real world or design
counterpart to the code. If you link your data manipulations to your data then
you make it hard to unlink data related by operation. If you link your
operations by related data, then you make it hard to unlink your operations
when the data changes or splits. If you keep your data in one place, write
operations in another place, and keep the aspects and roles of data intrinsic
from how the operations and transforms are applied to the data, then you will
find that many refactorings that would have been large and difficult in object
oriented code, become trivial. With this benefit comes a cost of keeping tabs
on what data is required for each operation, and the potential danger of
desynchronisation.  This consideration can lead to keeping some cold code in an
object oriented style where objects are responsible for maintaining internal
consistency over efficiency and mutability.

A big misunderstanding for many new to the data-oriented design paradigm,
a concept brought over from abstraction based development, is that we can
design a static library or set of templates to provide generic solutions to
everything presented in this book as a data-oriented solution. The awful truth
is that data, though it can be generic by type, is not generic in any real
world sense. The values are different, and often contain patterns that we can
turn to our advantage. How would it be possible to do compression if it were
not for patterns? Our runtime code can also benefit from this but it is
overlooked as being not object-oriented, or being too hard coded. It can be
better to hard code a transform than pretend it's not hard coded by wrapping it
in a generic container and using the wrong algorithms on it. Using existing
templates like this provide a known benefit of a minor increase in readability
to those who already know of the library, and potentially less bugs if the
functionality was in some way generic. But, if the functionality was not very
well mapped to the existing generic solution, writing it with a templated
function and then extending would possibly make the code harder to read by
hiding the fact that the technique had been changed subtly. Hard coding a new
algorithm is frequently a better choice as long as it has sufficient tests, and
tests are easier to write if you constrain yourself to the facts about the
data, and only work with simple data and not stateful objects.

\section{How is data formed?}

The games we write have a lot of data, in a lot of different formats. We have
textures in multiple formats for multiple platforms. There are animations,
usually optimised for different skeletons or types of playback. There are
sounds, lights, and scripts. Don't forget meshes, they now consist of multiple
buffers of attributes. Only a very small proportion of meshes are old fixed
function type with vertices containing positions, UVs, and normals. The data in
games development is hard to box, and getting harder to pin down as more ideas
that were previously considered impossible have now become commonplace. This is
why we spend a lot of time working on editors and tool chains, so we can take
the free form output from designers and artists and find a way to put it into
our engines. Without our tool-chains, editors, viewers, and tweaking tools,
there would be no way we could produce a game with the time we have. The
Object-oriented approach provides a good way to wrap our heads around all these
different formats of data. It gives a centralised view of where each type of
data belongs, and classifies it by what can be done to it. This makes it very
easy to add and use data quickly, but implementing all these different wrapper
objects takes time. Adding new functionality to these objects can sometimes
require large refactorings as occasionally objects are classified in such a way
that they don't allow for new features to exist. For example, in many old
engines, there was no way to upload a texture that didn't use a 32 bit or less
pixel stride. With the advent of floating point textures all that code required
a minor refactoring. In the past it was not possible to read a texture from the
vertex shader, so when texture based skinning came along, many engine
programmers had to refactor their render update. They had to allow for a vertex
shader texture upload because it might be necessary when uploading transforms
for rendering a skinned mesh. In many engines, mesh data is optimised for
rendering, but when you have to do mesh ray casting to see where bullets have
hit, or for doing IK, or physics, then you need multiple representations of an
entity, at which point the object oriented approach starts to look cobbled
together as there are less objects that represent real things, and more objects
used as containers so programmers can think in larger building blocks. These
blocks hinder though, as they become the only blocks used in thought, and stop
potential mental connections from happening. We went from 2D sprites, to 3D
meshes following the format of the hardware provider, to custom data streams
and compute units turning the streams into rendered triangles. Wave data, to
banks, to envelope controlled grain tables and slews of layered sounds. Tile
maps, to portals and rooms, to streamed, multiple level of detail chunks of
world, to hybrid mesh palette, props, and unique stitching assets. From
flip-book, to euler angle sequences, to quaternions and slerped anims, to
animation trees and behaviour mapping.

All these types of data are pretty common if you've worked in games at all, and
many engines do provide a abstraction to these more fundamental types. When a
new type of data becomes heavily used, it is promoted into the engine as a core
type. We normally consider the trade off of new types being handled as special
cases until they become ubiquitous, to be one of usability vs performance. We
don't want to provide free access to the lesser understood elements of game
development. People who are not, or can not, invest time in finding out how
best to use new features, should be discouraged from using them. The
Object-Oriented games development way to do that is to not provide objects that
represent them, and instead only offer the features to people who know how to
utilise the more advanced tools.

Apart from the objects representing digital assets, there are also objects for
internal game logic. For every game there are objects that only exist to
further the gameplay. Collectable card games have a lot of textures, but they
also have a lot objects for rules, card stats, player decks, match records,
even objects to represent the current state of play. All of these objects are
completely custom designed for one game. There may be sequels, but even they
could well use quite different game logic, and therefore require different
data, which would imply different methods on the now guaranteed different
objects.

Game data is complex. Any first layout of the data is inspired by the game's
design. However, once the development is underway, it needs to keep up with
however the game evolves. Object-oriented techniques offer a quick way to
implement a given design. Object-oriented development is quick at implementing
each design in turn, but it doesn't offer a way to migrate from one data schema
to the next. There are hacks, such as those used in version based asset
handlers, but normally, games developers change the tool-chain and the engine
at the same time, do a full re-export of all the assets, and commit to the next
version all in one go. This can be quite a painful experience if it has to
happen over multiple sites at the same time, or if you have a lot of assets, or
if you are trying to provide engine support for more than one title, and only
one wants to change to the new revision.

There has not yet been a successful effort to build a generic game asset
solution. This is because all games differ in so many subtle ways that if you
did provide a generic solution, it wouldn't be a game solution, just a new
language. There is no solution to be found in trying to provide all the
possible types of object that a game can use. But, there is a solution if we go
back to thinking about a game as merely running a set of computations on some
data.

\section[The framework]{What can provide a computational framework for such
complex data?}

Games developers are notorious for thinking about games development from either
a low level all out performance perspective, or from a very high level gameplay
and interaction perspective. This may have come about because of the widening
gap between the amount of code that has to be performant, and the amount of
code to make the game complete. Object-oriented techniques provide good
coverage of the high level aspect, and the assembler gurus have been hacking
away at performance for so long even their beards have beards. There has never
been much of a middle ground in games development, which is probably why the
structure and performance techniques employed by big-iron companies never
seemed useful. When games development was first flourishing in the late 1990's,
academic or corporate software engineering practices were seen as suspicious
because wherever they were employed, there was a dramatic drop in game
performance, and whenever any prospective employees came from those industries,
they never managed to impress. As games machines became more like standard
micro-computers, and standard micro-computers became more similar to the
mainframes of old, the more obvious it became that standard professional
software engineering practices could be useful. Now the scale of games has
grown to match the hardware, and with each successive generation, the number of
man hours to develop a game has grown exponentially, which is why project
management and software engineering practices have become de-facto at the
larger games companies. There was a time when games developers were seen as
cutting edge programmers, inventing new technology as the need arises, but with
the advent of less adventurous hardware (most notably in the x86 based XBox,
and it's only mildly interesting successor), there has been less call for
ingenious coding practices, and more call for a standardised process. This
means that game development can be tuned to ensure the release date will
coincide with marketing dates. There will always be an element of randomness in
high profile games development because there will always be an element of
innovation that virtually guarantees that you will not be able to predict how
long the project, or at least one part of the project, will take.

Part of the difficulty in adding new and innovative features to a game is the
data layout. If you need to change the data layout for a game, it will need
objects to be redesigned or extended in order to work within the existing
framework. If there is no new data, then a feature might require that
previously separate systems suddenly be able to talk to each other quite
intimately. This coupling can often cause system wide confusion with additional
temporal coupling and corner cases so obscure they can only be found one time
in a million. This sounds fine to some developers, but if you're expecting to
sell five to fifty million copies of your game, that's five to fifty people who
can take a video of your game, post it on the internet, and call your company
rubbish because they hadn't fixed an obvious bug.

Big iron developers had these same concerns back in the 1970's. Their software
had to be built to high standards because their programs would frequently be
working on data that was concerned with real money transactions. They needed to
write business logic that operated on the data, but most important of all, they
had to make sure the data was updated through a provably careful set of
operations in order to maintain its integrity. Database technology grew from
the need to process stored data, to do complex analysis on it, to store and
update it, and be able to guarantee that it was valid at all times. To do this,
the ACID test was used to ensured atomicity, consistency, isolation, and
durability.  Atomicity was the test to ensure that all transactions would
either complete, or do nothing. It could be very bad for a database to update
only one account in a financial transaction. There could be money lost, or
created if a transaction was not atomic. Consistency was added to ensure that
all the resultant state changes that should happen during a transaction do
happen, that is, all triggers that should fire, do, even if the triggers cause
triggers, with no limit. This would be highly important if an account should be
blocked after it has triggered a form of fraud detection. If a trigger has not
fired, then the company using the database could risk being liable for even
more than if they had stopped the account when they first detected fraud.
Isolation is concerned with ensuring that all transactions that occur cannot
cause any other transactions to differ in behaviour. Normally this means that
if two transactions appear to work on the same data, they have to queue up and
not try to operate at the same time. Although this is generally good, it does
cause concurrency problems. Finally, durability. This was the second most
important element of the four, as it has always been important to ensure that
once a transaction has completed, it remains so. In database terminology,
durability meant that the transaction would be guaranteed to have been stored
in such a way that it would survive server crashes or power outages. This was
important for networked computers where it would be important to know what
transactions had definitely happened when a server crashed or a connection
dropped.

Modern network games also have to worry about highly important data like this,
especially with non-free downloadable content, and even more so with consumable
downloadable content. To provide much of the functionality required of the
database ACID test, games developers have gone back to looking at how databases
were designed to cope with these strict requirements and found reference to
staged commits, idempotent functions, techniques for concurrent development and
a vast literature base on how to design tables for a database.

Databases provide a way to store highly complex data in a structured way while
providing a simple to learn language for transforming and generating reports
based on that data. The language, SQL, invented in the 1970's by Donald D.
Chamberlin and Raymond F. Boyce at IBM, provides a method by which it is
possible to store computable data while also maintaining data relationships.
But, games don't have simple computable data, they have classes and objects.
Database technology doesn't work for the Object-oriented approach that games
developers use.

The data in games can be highly complex, it doesn't neatly fit into database
rows. Not all objects can fit into rows of columns. Try to find the right table
columns to describe a level file. Try to find the right columns to describe a
car in a racing game, do you have a column for each wheel? Do you have a column
for each collision primitive, or just a column for the collision mesh?

The obvious answer is that game data just doesn't fit into the database way of
thinking. However, that's only because we've not normalised the data. We'll
stick with a level file for a while as we find out how years old techniques can
provide a very useful insight into what game data is really doing. There are
some types of data where it doesn't make sense to move into databases, such as
the raw assets (sounds, textures, vertex buffers etc.) but these can be seen as
primitives, much like the integers, floating point numbers, strings and boolean
values we shall be working with.

What we shall discover is that everything we did was already in a database, but
it wasn't obvious to us because of how we store our data. The structure of our
data is a trade-off against performance, readability, maintenance, future
proofing, extendibility and reuse. The most flexible database in common use is
the file-system. It has one table with two columns. A primary key of the file
path, and a string for the data. This simplest database system is the perfect
fit for a completely future proof system. The more complex the tables get, the
less future proof, and the less maintainable, but the higher the performance
and readability. For example, a file has no documentation of its own, but the
schema of a database could be all that is required to understand a sufficiently
well designed database. That's how games don't even appear to have databases.
They are so complex for the sake of performance that they have forgotten that
they are merely a data transform.


